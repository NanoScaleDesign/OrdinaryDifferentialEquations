\newpage
\section{Coursework}
The aim of this coursework is to give you the opportunity to think more about applications of ODE's and give you the freedom to follow your interest.

Your task is as follows:
Imagine that you are the teacher of this course.
Considering any application involving ODE's of your choice, create a short document explaining to your students about the application and how ODE's are involved.
Create one or two challenges for your peers to test their understanding of the topic after reading your document.
You may use ODE mathematics that have been covered in the course, or you may use something new.
Points will be awarded on the basis of demonstration of knowledge, quality and depth of analysis and explanation, accuracy and suitable level of pitch.

Guidance:

\textbf{1)} Please include equations, figures and references as appropriate. Especially references.

\textbf{2)} Create 1-2 challenges to accompany your report, so someone reading your document can test their knowledge.

\textbf{3)} Include \textbf{fully worked} solutions to challenges you make (ie, not only the final answer, but clearly show the steps involved in order to achieve the final answer).

\textbf{4)} Pitch the document at a level where your peers in class can read and understand it. This is important because we may use the reports as a basis for study in the final class.

\textbf{5)} Aim for 2-3 pages in length, including references and challenge questions. The maximum allowed length is 4 pages, excluding worked solutions to your challenges. Your worked-solutions may cover as many pages as you require in order to explain the calculation steps required.

\subsection{Submission}
You must submit \textbf{both a paper and electronic version}. Submit the materials by \textbf{email} to the teacher by \textbf{10:00 on 26 January 2018} with the subject ``[ODE] Coursework'' and \textbf{bring a paper copy to the class on that day}.

The electronic version may be in any format, including LibreOffice, MS Word, Google docs, Latex, etc\ldots If you submit a PDF, please also submit the source-files used to generate the PDF.

Late submission:\\
By 10:00 on 27 January 2018 (electronic submission only): 90\% of the final mark.\\
By 10:00 on 2 February 2018: 50\% of the final mark.\\
Later submissions cannot be considered.

Good luck.

\subsection{Questions about the coursework}

I found out that the report that I want to write requires some mathematics about [..] which is unrelated to ODE's, and it is difficult to understand. Do I need to include calculations about how [..] works or should I just use the result in my discussion about ODE's?

\emph{The coursework is designed to focus on your understanding of ODE's, and therefore this is where you should focus your discussion. If you need to bring in results from other branches of mathematics you're welcome to do so and it is not necessary to demonstrate deep understanding of such mathematics.}

I am interested to study about [..] which is an ODE topic, but not something that we covered in the course. Can I still study it?

\emph{Yes! There are many things that we could have covered but I chose not to, simply due to lack of time. You're welcome to choose a topic like that.}

I am interested to study about [..] but I'm worried it will be too hard to write a good report. Should I do something else?

\emph{In terms of difficulty, I can appreciate that some topics will be more difficult than others. Also, it can be difficult to know in advance how difficult the topic is going to be. But I will take this into account. So for the same grade, I would expect a relatively simple topic to be explored more deeply, and for a harder topic to be explored less deeply.}

