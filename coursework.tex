\newpage
\section{Coursework}
Ordinary differential equations arise in a wide range of situations. This coursework is designed to give you the opportunity to investigate an application or phenomenon related to your field of interest that involves the use of ODE's.

The task is as follows:

\textbf{1)} Write a report at least 1 full page in length, explaining about an application or phenomenon which can be described in terms of Ordinary Differential Equations. Please include equations, figures and references.

\textbf{2)} Create at least 2 challenges to accompany your report, so someone reading your document can test their knowledge.

\textbf{3)} Include solutions to the challenges you make.

I may (or may not) choose to incorporate some aspects of the submissions into teaching of the final 1 or 2 classes.

\subsection{Submission}
You must submit \textbf{both a paper and electronic version}. Submit the materials by \textbf{email} to the teacher by \textbf{10:30 on 12 January 2017} with the subject ``[ODE] Coursework'' and \textbf{bring a paper copy to the class on that day}.

The electronic version may be in any format, including LibreOffice, MS Word, Google docs, Latex, etc\ldots If you submit a PDF, please also submit the source-files used to generate the PDF.

Late submission:\\
By 10:00 on 13 January 2017 (electronic submission only): 90\% of the final mark.\\
By 10:00 on 16 January 2017: 50\% of the final mark.\\
Later submissions cannot be considered.

\subsection{Marking}
Marks will be assigned based on
\begin{itemize}
    \item Relevance: An application or phenomenon that has a basis in ODE's.
    \item Originality: It should be your own work. You must \textbf{cite all references, as well as images and text taken from other sources}.
    \item Level: The subject should be pitched at a level whereby anyone else in the class could learn about the subject based on your work.
    \item Explanation: An accurate explanation described with depth and clarity.
\end{itemize}

