\documentclass[a4paper]{book} %{article}

\usepackage{fullpage} % Package to use full page
\usepackage{parskip} % Package to tweak paragraph skipping
\usepackage{tikz} % Package for drawing
\usepackage{amsmath}
\usepackage{hyperref}
\usepackage[numbered]{bookmark} % For numbering of challenges in bookmark pane of PDF viewer
\def\UrlBreaks{\do\/\do-}
\usepackage[absolute]{textpos}
\setlength{\TPHorizModule}{1mm}
\setlength{\TPVertModule}{1mm}
\usepackage{tikz}
\usepackage{siunitx}
\usepackage{datetime} % Time
\usepackage[UKenglish]{isodate}
\usepackage{ctable} % Thick table lines
\usepackage{booktabs} % Merge cells with multicolumn
\usepackage{bm} % Bold-maths \bm command
\usepackage{pdfpages} % Insert PDF into PDF

\newcommand{\courseyear}{2017 }
\newcommand{\disctime}{14:50 to 16:20 }
\newcommand{\discdays}{Fridays }
\newcommand{\discroom}{Centre Zone 1409}
\newcommand{\course}{Ordinary Differential Equations }
\newcommand{\courseurl}{ordinary-differential-equations}
\newcommand{\nensei}{2nd}

\newcommand{\lap}[1]{\mathcal{L}\{#1\}}

\newcommand{\six}[1]{\SI[parse-numbers=false]{X}{#1}}
\newcommand{\hash}[2]{MD5(#1\_X) = #2\ldots}
\newcommand{\timebox}{\vfill Study-time (from end of previous challenge to end of this challenge): \underline{\hspace{1cm} minutes}}
\newcommand{\matrixcrr}[2]{\left(\begin{array}{c}{#1}\\{#2}\end{array}\right)}
\newcommand{\matrixccrr}[4]{\left(\begin{array}{cc}{#1}&{#2}\\{#3}&{#4}\end{array}\right)}

\newcommand{\solint}[2]{X = Your solution\\Form: Integer.\\Place the indicated letter in front of the number.\\Example: aX where $X=46$ is entered as \href{http://www.wolframalpha.com/input/?i=md5+hash+of+\%22a46\%22}{a46}\\\\Hash of {#1}X = {#2}}
\newcommand{\soltwodp}[2]{X = Your solution\\Form: Decimal to 2 decimal places.\\Place the indicated letter in front of the number.\\Example: aX where $X=46.00$ is entered as \href{http://www.wolframalpha.com/input/?i=md5+hash+of+\%22a46.00\%22}{a46.00}\\\\Hash of {#1}X = {#2}}
\newcommand{\solstr}[2]{X = Your solution\\Form: String.\\Place the indicated letter in front of the string.\\Example: aX where $X=\text{abcdef}$ is entered as \href{http://www.wolframalpha.com/input/?i=md5+hash+of+\%22a46.00\%22}{aabcdef}\\\\Hash of {#1}X = {#2}}

\graphicspath{{Images/}}

\begin{document}

\begin{titlepage}
    \begin{center}
        \vspace*{1cm}

        \Huge
        \textbf{Ordinary Differential Equations}

        Autumn \courseyear

        \vspace{1.5cm}
        \Large
        Last updated:\\\today \ at \currenttime

        \vspace{4.0cm}
        \LARGE
        James Cannon\\Kyushu University
        \vfill

        \normalsize
        \url{http://www.jamescannon.net/teaching/\courseurl}\\
        \vspace{0.3cm}
        \small
        \url{http://raw.githubusercontent.com/NanoScaleDesign/OrdinaryDifferentialEquations/master/ode.pdf}
        \vspace{0.5cm}

        License: \emph{CC BY-NC 4.0}.

    \end{center}
\end{titlepage}

\setcounter{chapter}{-1}

\tableofcontents

\chapter{Course information}
\newpage
\input{instructions}
\newpage
\section{Timetable}

\begin{center}
    \begin{tabular}{|c|c|c|c|}
        \hline
        & \textbf{Discussion} & \textbf{Target} & \textbf{Note} \\ \specialrule{.1em}{.05em}{.05em}
        \textbf{1}  & 7 Oct  & -            &                          \\ \hline
        \textbf{2}  & 14 Oct & 2.2          &                          \\ \hline
        \textbf{3}  & 21 Oct & 2.7          &                          \\ \hline
        \textbf{4}  & 28 Oct & 2.12         &                          \\ \specialrule{.1em}{.05em}{.05em}
        \textbf{5}  & 4 Nov  & 2.20         &                          \\ \hline
        \textbf{6}  & 11 Nov & 3.8          &                          \\ \hline
        \textbf{7}  & 25 Nov & 3.15         &                          \\ \specialrule{.1em}{.05em}{.05em}
        \textbf{8}  & 2 Dec  & 3.18         & Coursework instructions  \\ \hline                            % Non-homogeneous equations (undetermined coeffs, var of params)
        \textbf{9}  & 9 Dec  & Midterm exam &                          \\ \hline                            % Exam preparation
        \textbf{10} & 16 Dec & 4.6          &                          \\ \specialrule{.1em}{.05em}{.05em}  % Laplace transform
        \textbf{11} & 6 Jan  &              &                          \\ \hline                            % Laplace transform
        \textbf{12} & 12 Jan &              & Submission of coursework \\ \hline                            % Systems of differential equations
        \textbf{13} & 20 Jan &              &                          \\ \hline                            % Systems of differential equations
        \textbf{14} & 27 Jan &              &                          \\ \specialrule{.1em}{.05em}{.05em}  % Series
        \textbf{15} & 10 Feb & Final exam   &                          \\ \hline
    \end{tabular}
\end{center}

Example: To keep pace with the course, you should aim to complete challenge 2 of chapter 2 by the 14th of October.

\input{hash}
%\newpage
\section{Coursework}
The aim of this coursework is to give you the opportunity to think more about applications of ODE's and give you the freedom to follow your interest.

Your task is as follows:
Imagine that you are the teacher of this course.
Considering any application involving ODE's of your choice, create a short document explaining to your students about the application and how ODE's are involved.
Create one or two challenges for your peers to test their understanding of the topic after reading your document.
You may use ODE mathematics that have been covered in the course, or you may use something new.
Points will be awarded on the basis of demonstration of knowledge, quality and depth of analysis and explanation, accuracy and suitable level of pitch.

Guidance:

\textbf{1)} Please include equations, figures and references as appropriate. Especially references.

\textbf{2)} Create 1-2 challenges to accompany your report, so someone reading your document can test their knowledge.

\textbf{3)} Include \textbf{fully worked} solutions to challenges you make (ie, not only the final answer, but clearly show the steps involved in order to achieve the final answer).

\textbf{4)} Pitch the document at a level where your peers in class can read and understand it. This is important because we may use the reports as a basis for study in the final class.

\textbf{5)} Aim for 2-3 pages in length, including references and challenge questions. The maximum allowed length is 4 pages, excluding worked solutions to your challenges. Your worked-solutions may cover as many pages as you require in order to explain the calculation steps required.

\subsection{Submission}
You must submit \textbf{both a paper and electronic version}. Submit the materials by \textbf{email} to the teacher by \textbf{10:00 on 26 January 2018} with the subject ``[ODE] Coursework'' and \textbf{bring a paper copy to the class on that day}.

The electronic version may be in any format, including LibreOffice, MS Word, Google docs, Latex, etc\ldots If you submit a PDF, please also submit the source-files used to generate the PDF.

Late submission:\\
By 10:00 on 27 January 2018 (electronic submission only): 90\% of the final mark.\\
By 10:00 on 2 February 2018: 50\% of the final mark.\\
Later submissions cannot be considered.

Good luck.


% NT: Add hash-practise\ldots
\chapter{Hash practise}
\section{Hash practise: Integer}

X = 46.3847\\
Form: Integer.\\
Place the indicated letter in front of the number.\\
Example: aX where $X=46$ is entered as \href{http://www.wolframalpha.com/input/?i=md5+hash+of+\%22a46\%22}{a46}

hash of aX = e77fac

\section{Hash practise: Decimal}

X = 49\\
Form: Two decimal places.\\
Place the indicated letter in front of the number.\\
Example: aX where $X=46.00$ is entered as \href{http://www.wolframalpha.com/input/?i=md5+hash+of+\%22a46.00\%22}{a46.00}

hash of bX = 82c9e7

\section{Hash practise: String}

X = abcdef\\
Form: String.\\
Place the indicated letter in front of the number.\\
Example: aX where $X=abc$ is entered as \href{http://www.wolframalpha.com/input/?i=md5+hash+of+\%22aabc\%22}{aabc}

hash of cX = 990ba0

\section{Hash practise: Scientific form}

X = 500,765.99\\
Form: Scientific notation with the mantissa in standard form to 2 decimal place and the exponent in integer form.\\
Place the indicated letter in front of the number.\\
Example: aX where $X=4 \times 10^{-3}$ is entered as \href{http://www.wolframalpha.com/input/?i=md5+hash+of+\%22a4.00e-3\%22}{a4.00e-3}

hash of dX = be8a0d


\chapter{Definitions}
\section{Order of a differential equation}

\subsection*{Resources}
\begin{itemize}
    \item Text: \url{http://tutorial.math.lamar.edu/Classes/DE/Definitions.aspx}
\end{itemize}

\subsection*{Challenge}
What is the sum of the orders of the following equations?

\begin{equation}
    \frac{dy}{dx}A = 5x^3 + 3
\end{equation}

\begin{equation}
    cos(y) y'''(x) - y(x) = 25
\end{equation}

\begin{equation}
    \frac{d}{dx} \frac{d^2 y}{dx^2} = \frac{x^{-2}}{3}
\end{equation}

\subsection*{Solution}
X = Your solution\\
Form: Integer\\
Place the indicated letter in front of the number\\
Example: aX where $X=46$ is entered as \href{http://www.wolframalpha.com/input/?i=md5+hash+of+\%22a46\%22}{a46}

hash of eX = 492585




%%%%%%%%%%%%%%%%%%%%%%%%%%%%%%%%%
\newpage
%%%%%%%%%%%%%%%%%%%%%%%%%%%%%%%%%
\section{Identifying linear and non-linear differential equations}

\subsection*{Comment}
Being able to identify linear and non-linear ODE's will help you understand how to approach different problems.

A function is linear if it satisfies the following condition for all $y_1$ and $y_2$:
\begin{equation}
    \label{eq:lindef}
    f(y_1 + y_2)  = f(y_1) + f(y_2)
\end{equation}

For example,
\begin{equation*}
    y = 5x^2 + 4
\end{equation*}
is trivally linear, because setting $f(y)$ to be equal to the LHS (which is only a function of $y$),
\begin{equation*}
    f(y_1 + y_2) = (y_1+y_2) = y_1 + y_2 = f(y_1) + f(y_2)
\end{equation*}

The function on the left side does not have to be so simple. It just has to be a function of $y$. For example, the equation
\begin{equation*}
    y'' - 4yx = ln x - y
\end{equation*}
is more complex. However it is still linear. Rearranging to collect all the $y$-terms together:
\begin{equation*}
    f(y) = y'' - 4yx + y = ln x
\end{equation*}
linearity can be proved:
\begin{align*}
    f(y_1 + y_2) &= (y_1 + y_2)'' - 4(y_1 + y_2)x + (y_1 + y_2) \\
                 &= y_1''+ y_2'' - 4y_1x + 4y_2x + y_1 + y_2 \\
                 &= y_1 + 3y_1x + y_1 + y_2 + 3y_2x + y_2 \\
                 &= f(y_1) + f(y_2)
\end{align*}

For non-linear equations however, equation \ref{eq:lindef} does not hold true. For example,
\begin{equation*}
    5 + yy' = x - y
\end{equation*}
can be rearranged to isolate all the y-terms on the LHS as
\begin{equation*}
    f(y) = yy' + y = x - 5
\end{equation*}
however it can easily be seen than \ref{eq:lindef} does not hold true:
\begin{align*}
    f(y_1 + y_2) &= (y_1 + y_2)(y_1+y_2)' + (y_1 + y_2) \\
                 &= y_1 y_1' + y_2 y_2' + y_1 y_2' + y_2 y_1' + y_1 + y_2 \\
                 &= y_1 y_1' + y_1 + y_2 y_2' + y_2 + y_1 y_2' + y_2 y_1' \\
                 &= f(y_1) + f(y_2) + y_1 y_2' + y_2 y_1'
\end{align*}
which does not satisfy equation \ref{eq:lindef} for all $y_1$ and $y_2$.

\subsection*{Challenge}
Sum the points corresponding to the equations that are linear. You may be able to judge some by eye, but you should prove mathematically that at least one of the equations are linear and at least one of the equations are non-linear.

1 point: $\displaystyle \frac{dx}{dt} = 5t^3 + 3$.

2 points: $\displaystyle cos(x) x'''(t) - x(t) = 25$.

4 points: $\displaystyle \frac{d}{dt} \frac{d^2 x}{dt^2} = \frac{t^{-2}}{3}$.

8 points: $\displaystyle x'(t) - sin(x(t)) = 0$.

16 points: $\displaystyle x'(t) - x(t) = 0$.

32 points: $\displaystyle t x'(t) - x(t) = 0$.

\subsection*{Solution}
X = Your solution\\
Form: Integer\\
Place the indicated letter in front of the number\\
Example: aX where $X=46$ is entered as \href{http://www.wolframalpha.com/input/?i=md5+hash+of+\%22a46\%22}{a46}

hash of rX = f5d2c0




%%%%%%%%%%%%%%%%%%%%%%%%%%%%%%%%%
\newpage
%%%%%%%%%%%%%%%%%%%%%%%%%%%%%%%%%
\section{Linear differential equations vs non-linear differential equations}

\subsection*{Resources}
\begin{itemize}
    \item Wikipedia: \url{https://en.wikipedia.org/wiki/Nonlinear_system#Nonlinear_differential_equations}
    \item Wikipedia: \url{https://en.wikipedia.org/wiki/Linear_differential_equation}
\end{itemize}

\subsection*{Challenge}
Write no-more than 1 short paragraph describing in qualitative terms the difference between a linear and non-linear differential equation.

\subsection*{Solution}
Please compare with your partner in class and discuss with the teacher if you are unsure.




%%%%%%%%%%%%%%%%%%%%%%%%%%%%%%%%%
\newpage
%%%%%%%%%%%%%%%%%%%%%%%%%%%%%%%%%

\section{Valid solutions}

\subsection*{Resources}
\begin{itemize}
    \item Text: \url{http://tutorial.math.lamar.edu/Classes/DE/Definitions.aspx}
\end{itemize}

\subsection*{Challenge}

Use substitution to prove that

\begin{equation}
    y=\frac{5}{5+x}
\end{equation}

is a solution to the equation

\begin{equation}
    x y'+y=y^2
\end{equation}

and state the value of $x$ for which the solution is undefined.

\subsection*{Solution}
Value of $x$ for which solution is undefined:

\soltwodp{t}{829f33}




%%%%%%%%%%%%%%%%%%%%%%%%%%%%%%%%%
\newpage
%%%%%%%%%%%%%%%%%%%%%%%%%%%%%%%%%

\section{Range of valid solutions}

\subsection*{Resources}
\begin{itemize}
    \item Text: \url{http://tutorial.math.lamar.edu/Classes/DE/Definitions.aspx}
\end{itemize}

\subsection*{Challenge}

Use substitution to prove that

\begin{equation}
    y = -\sqrt{100-x^2}
\end{equation}

is a solution to the equation

\begin{equation}
    x + y y' = 0
\end{equation}

and state the range of x for which the solution is valid. Enter the value of the lower range as the solution below.

\subsection*{Solution}
\soltwodp{y}{d96920}

% In addition, cover initial conditions
\chapter{1st-order differential equations}
\section{Determining a simple DE from a description}

\subsection*{Resources}
\begin{itemize}
    \item Text: \url{http://tutorial.math.lamar.edu/Classes/DE/Definitions.aspx}
\end{itemize}

\subsection*{Challenge}
Newton's law of cooling states that the rate of cooling of an object is proportional to the temperature difference with the ambient surroundings. (a) Write a differential equation describing this situation. (b) Assuming a proportionality constant of \num{0.2} \si{/hour}, what is the rate of temperature change when the object is at \SI{30}{\degreeCelsius} and the ambient temperature is \SI{20}{\degreeCelsius}?

\subsection*{Solution}
(units: \si{\degreeCelsius\per\hour})

\soltwodp{q}{4aca8d}




%%%%%%%%%%%%%%%%%%%%%%%%%%%%%%%%%
\newpage
%%%%%%%%%%%%%%%%%%%%%%%%%%%%%%%%%
\section{Direction (Slope) fields}

\subsection*{Resources}
\begin{itemize}
    \item Text: \url{http://tutorial.math.lamar.edu/Classes/DE/DirectionFields.aspx}
    \item Video 1: \url{https://www.khanacademy.org/math/differential-equations/first-order-differential-equations/differential-equations-intro/v/creating-a-slope-field}
    \item Video 2: \url{https://www.khanacademy.org/math/differential-equations/first-order-differential-equations/differential-equations-intro/v/slope-field-to-visualize-solutions}
\end{itemize}

\subsection*{Comment}
It is good practise to try drawing the below fields before looking at the next page. You need to be able to go in both directions (ie, drawing and recognising). You will not be given a glimps at the fields in the exam prior to being asked to draw them.

\subsection*{Question}
Try drawing the slope field for at least 3 of the equations given below (your choice). Then, put the slope fields given on the next page in the same order as these equations.

\begin{enumerate}
    \item $y'=x$
    \item $y'=0.2y$
    \item $y'=0.2y(1-y/6)$
    \item $y'=(x-y)/(x+y)$
    \item $y'=2(y-1)/x$
    \item $y'=2y/(x+5)$
\end{enumerate}

\newpage

\includegraphics{direction_fields_A.png}
\includegraphics{direction_fields_B.png}
\includegraphics{direction_fields_C.png}

\includegraphics{direction_fields_D.png}
\includegraphics{direction_fields_E.png}
\includegraphics{direction_fields_F.png}

\subsection*{Solution}
\solstr{q}{e93bfe}



\iffalse

%%%%%%%%%%%%%%%%%%%%%%%%%%%%%%%%%
\newpage
%%%%%%%%%%%%%%%%%%%%%%%%%%%%%%%%%
\section{Solving a simple 1st-order linear equation}

\subsection*{Resources}
\begin{itemize}
    \item Video: \url{https://www.khanacademy.org/math/differential-equations/first-order-differential-equations/differential-equations-intro/v/finding-particular-linear-solution-to-differential-equation}
\end{itemize}

\subsection*{Comment}
It's not easy to see when equations can be solved simply, like in the challenge below, and when they can not. But for a 1st-order linear differential equation, this method is often good to try first. If it's not 1st order and not linear, you know to try a different approach.

\subsection*{Challenge}
Determine the value of $y(x=1)$ for the following equation:

\begin{equation}
    y'=2y-5x+2
\end{equation}

\subsection*{Solution}
\six{}

\hash{u}{505c7b}




%%%%%%%%%%%%%%%%%%%%%%%%%%%%%%%%%
\newpage
%%%%%%%%%%%%%%%%%%%%%%%%%%%%%%%%%
\section{Separable equations I}

\subsection*{Resources}
\begin{itemize}
    \item Video I: \url{https://www.khanacademy.org/math/differential-equations/first-order-differential-equations/separable-equations/v/separable-differential-equations-introduction} 
    \item Video II: \url{https://www.khanacademy.org/math/differential-equations/first-order-differential-equations/separable-equations/v/particular-solution-to-differential-equation-example}
    \item Text:\url{http://tutorial.math.lamar.edu/Classes/DE/Separable.aspx}
\end{itemize}

\subsection*{Challenge}
Given the following equation:
\begin{equation}
    r' = -Sin(\theta)
\end{equation}
Determine the function $r(\theta)$ that passes through the point (0,1) in $\theta-r$ space, and then solve for $\theta = \pi/4$.


\subsection*{Solution}
\six{}

\hash{i}{286117}




%%%%%%%%%%%%%%%%%%%%%%%%%%%%%%%%%
\newpage
%%%%%%%%%%%%%%%%%%%%%%%%%%%%%%%%%
\section{Separable equations II}

\subsection*{Resources}
\begin{itemize}
    \item Video I: \url{https://www.khanacademy.org/math/differential-equations/first-order-differential-equations/separable-equations/v/separable-differential-equations-introduction} 
    \item Video II: \url{https://www.khanacademy.org/math/differential-equations/first-order-differential-equations/separable-equations/v/particular-solution-to-differential-equation-example}
    \item Text:\url{http://tutorial.math.lamar.edu/Classes/DE/Separable.aspx}
\end{itemize}

\subsection*{Challenge}
Given the following equation:
\begin{equation}
    r' cot(\theta) + r = 2
\end{equation}
Determine the function $r(\theta)$ that passes through the point (0,1) in $\theta-r$ space, and then solve for $\theta = \pi/4$.

\subsection*{Solution}
\six{}

\hash{o}{87ee92}




%%%%%%%%%%%%%%%%%%%%%%%%%%%%%%%%%
\newpage
%%%%%%%%%%%%%%%%%%%%%%%%%%%%%%%%%
\section{Separable equations III}

\subsection*{Resources}
\begin{itemize}
    \item Video I: \url{https://www.khanacademy.org/math/differential-equations/first-order-differential-equations/separable-equations/v/separable-differential-equations-introduction} 
    \item Video II: \url{https://www.khanacademy.org/math/differential-equations/first-order-differential-equations/separable-equations/v/particular-solution-to-differential-equation-example}
    \item Text:\url{http://tutorial.math.lamar.edu/Classes/DE/Separable.aspx}
\end{itemize}

\subsection*{Challenge}
Given the following equation:
\begin{equation}
    y' x^2 - y = 0
\end{equation}
Determine the function $y(x)$ that passes through the point $x=2,y=1$ and then solve for the given $x$ value. State the value of $x$ where the solution is undefined. % Would be nice to couple this with a field graph.

\subsection*{Solution}
Solve for $x=4$:

\six{}

\hash{p}{7cb08e} 

Value of x where solution is undefined:

\six{}

\hash{a}{b30fe7} 




\stepcounter{section}




%%%%%%%%%%%%%%%%%%%%%%%%%%%%%%%%%
\newpage
%%%%%%%%%%%%%%%%%%%%%%%%%%%%%%%%%
\section{Logistic equation}

\subsection*{Resources}
\begin{itemize}
    \item Videos: The 5 videos on the logistic differential equation and function starting at: \url{https://www.khanacademy.org/math/differential-equations/first-order-differential-equations/logistic-differential-equation/v/modeling-population-with-differential-equations}
\end{itemize}

\subsection*{Comment}
The rate of growth can be calculated considering the equation $dN/dt=rN$. This was not clear in an earlier version of the challenge. The question has also been adjusted (100 mg instead of 300 mg) to minimise the chance of rounding errors effecting final answer. The hash-code has been updated to reflect this.

\subsection*{Challenge}
Assuming there is no-limit on growth, a given bacteria would be able to reproduce at such a rate that the amount of bacteria measured in mg increases by 20\% every 25 hours. However, due to environmental factors the limiting (maximum) amount of bacteria that can exist in the system at any one time is 400 mg. Assuming an initial amount of bacteria of 20 mg, how much time, rounded to the nearest integer hours, must one wait to reach 100 mg of bacteria?

\emph{Note: Be sure to maintain a sufficient number of significant figures in your numbers while performing the calculation.}

\subsection*{Solution}
\six{hours} (expressed as an integer - do not enter ``.00''.)

\hash{d}{0eba84} 




%%%%%%%%%%%%%%%%%%%%%%%%%%%%%%%%%
\newpage
%%%%%%%%%%%%%%%%%%%%%%%%%%%%%%%%%
\section{Autonomous differential equations}

\subsection*{Resources}
\begin{itemize}
    \item Text: \url{http://tutorial.math.lamar.edu/Classes/DE/EquilibriumSolutions.aspx}
    \item Wikipedia: \url{https://en.wikipedia.org/wiki/Autonomous_system_(mathematics)}
\end{itemize}

\subsection*{Challenge}
Add the points of the autonomous differential equations in the following list:

1 point: $y' = cos(y)-5$

2 points: $y' = cos(y)/x - 5$

4 points: $y' = cos(y)/x - 5/x$

8 points: $y^2 = y' y+5$

16 points: $x y' = 5 y$

32 points: $y' = 1$

\subsection*{Solution}
\six{}

\hash{f}{9bf043} 
% Would be good to add some vector fields to show that it's independent of x.




%%%%%%%%%%%%%%%%%%%%%%%%%%%%%%%%%
\newpage
%%%%%%%%%%%%%%%%%%%%%%%%%%%%%%%%%
\section{The stability of solutions I}

\subsection*{Resources}
\begin{itemize}
    \item Text: \url{http://tutorial.math.lamar.edu/Classes/DE/EquilibriumSolutions.aspx}
    \item Text: \url{http://www.math.psu.edu/tseng/class/Math251/Notes-1st\%20order\%20ODE\%20pt2.pdf}
\end{itemize}

\subsection*{Challenge}
Considering the logistic equation $N'=0.2N(1-N/6)$, make 3 separate lists containing any equilibrium, semi-stable and unstable y-values.

To check your answer, sum the value of each list. If there are no values in a list, simply enter ``none'' to check the result.

\subsection*{Solution}
\subsubsection*{Stable}
\six{}

\hash{g}{2c32d8} 

\subsubsection*{Semi-stable}
\six{}

\hash{h}{8b595d} 

\subsubsection*{Unstable}
\six{}

\hash{j}{4fd3f6}




%%%%%%%%%%%%%%%%%%%%%%%%%%%%%%%%%
\newpage
%%%%%%%%%%%%%%%%%%%%%%%%%%%%%%%%%
\section{The stability of solutions II}

\subsection*{Resources}
\begin{itemize}
    \item Text: \url{http://tutorial.math.lamar.edu/Classes/DE/EquilibriumSolutions.aspx}
    \item Text: \url{http://www.math.psu.edu/tseng/class/Math251/Notes-1st\%20order\%20ODE\%20pt2.pdf}
\end{itemize}

\subsection*{Challenge}
Considering the differential equation $y'=(y^2-16)(y+3)^2$, make 3 separate lists containing any equilibrium, semi-stable and unstable y-values.

To check your answer, sum the value of each list. If there are no values in a list, simply enter ``none'' to check the result.

\subsection*{Solution}
\subsubsection*{Stable}
\six{}

\hash{k}{667798} 

\subsubsection*{Semi-stable}
\six{}

\hash{z}{200aa3} 

\subsubsection*{Unstable}
\six{}

\hash{x}{d1ee5b}




%%%%%%%%%%%%%%%%%%%%%%%%%%%%%%%%%
\newpage
%%%%%%%%%%%%%%%%%%%%%%%%%%%%%%%%%
\section{Euler's method}

\subsection*{Resources}
\begin{itemize}
    \item Videos and exersizes in the ``Euler's Method'' section of Khan academy: \url{https://www.khanacademy.org/math/differential-equations/first-order-differential-equations/eulers-method-tutorial/v/eulers-method}
    \item Text: \url{http://tutorial.math.lamar.edu/Classes/DE/EulersMethod.aspx}
\end{itemize}

\subsection*{Challenge}
Considering the differential equation $y'=10-y$, an initial value of $y(0)=1$ and a step size of $\Delta x = 0.2$, use Euler's method to estimate the value of $y(x=1)$. The actual solution, $y(x)=10-9e^{-x}$, is shown below.

\includegraphics{eulers_method.png}

\subsection*{Solution}
\six{}

\hash{c}{1f90fa} 
% It would be nice to see if this error actually goes to zero in the end.




%%%%%%%%%%%%%%%%%%%%%%%%%%%%%%%%%
\newpage
%%%%%%%%%%%%%%%%%%%%%%%%%%%%%%%%%
\section{Exact differential equations: derivation}

\subsection*{Resources}
\begin{itemize}
    \item Videos: \url{https://www.khanacademy.org/math/differential-equations/first-order-differential-equations/exact-equations/v/exact-equations-intuition-1-proofy}
\end{itemize}

\subsection*{Challenge}
Please follow the two videos on derivation and intuition regarding exact differential equations starting at the video listed above.

If
\begin{equation}
    \frac{d \psi(x,y)}{dx} = 2xy + x^2y' - (x+y)/100
\end{equation}

what is $\displaystyle \frac{\partial \psi}{\partial x}$?

To check your answer, substitute $x=3.1$ and $y=-2$ into the resulting equation.

\subsection*{Solution}
\six{}

\hash{v}{f7f178}
% Note that video is not so clear about psi being a full rather than partial derivative. This question should help address this problem.




%%%%%%%%%%%%%%%%%%%%%%%%%%%%%%%%%
\newpage
%%%%%%%%%%%%%%%%%%%%%%%%%%%%%%%%%
\section{Exact differential equations: possible solutions given $\psi_x$}

\subsection*{Resources}
\begin{itemize}
    \item Videos: Exact equations intuition 1,2 and examples 1,2,3 starting from \url{https://www.khanacademy.org/math/differential-equations/first-order-differential-equations/exact-equations/v/exact-equations-intuition-1-proofy}
    \item Text: \url{http://tutorial.math.lamar.edu/Classes/DE/Exact.aspx}
\end{itemize}

\subsection*{Challenge}
Sum the points of all the possible solutions to the integral of the partial-differential equation:
\begin{equation}
    \psi_x = 6x - 3e^x sin(y)
\end{equation}

1 point: $\psi(x,y) = 3x^2-3e^x sin(y) + 4$

2 points: $\psi(x,y) = 3x^2-3e^x sin(y) + x$

4 points: $\psi(x,y) = 3x^2-3e^x sin(y) + y$

8 points: $\psi(x,y) = 3x^2-3e^x sin(y) + yx$

16 points: $\psi(x,y) = 3x^2-3e^x sin(y) + y^2$

32 points: $\psi(x,y) = 3x^2-3e^x sin(y) + 5 sin(y)$

64 points: $\psi(x,y) = 3x^2-3e^x sin(y) + 5 sin(y)cos(x)$

\subsection*{Solution}
\six{}

\hash{b}{408993} 




%%%%%%%%%%%%%%%%%%%%%%%%%%%%%%%%%
\newpage
%%%%%%%%%%%%%%%%%%%%%%%%%%%%%%%%%
\section{Exact differential equations: identification}
\label{sec:edeid}

\subsection*{Resources}
\begin{itemize}
    \item Videos: Exact equations intuition 1,2 starting from \url{https://www.khanacademy.org/math/differential-equations/first-order-differential-equations/exact-equations/v/exact-equations-intuition-1-proofy}
    \item Text: \url{http://tutorial.math.lamar.edu/Classes/DE/Exact.aspx}
\end{itemize}

\subsection*{Challenge}
Sum the points of the equations below that are exact differential equations:

1 point: $\displaystyle (3x^2y+8xy^2) dx + (x^3 + 8x^2y + 12 y^2) dy = 0$ % A

2 points: $\displaystyle sin(x) cos(y) dx + cos(x) sin(y) dy = 0$ % B

4 points: $\displaystyle sin(x) cos(y) dx + sin(x) sin(y) dy = 0$ % C

8 points: $\displaystyle \frac{dx}{x} + \frac{dy}{y} = 0$ % D

16 points: $\displaystyle -\frac{y dx + x dy}{x^2} = 0 $ % E

32 points: $\displaystyle -\frac{y dx - x dy}{x^2} = 0$ % F


\subsection*{Solution}
\six{}

\hash{n}{868f48} 




%%%%%%%%%%%%%%%%%%%%%%%%%%%%%%%%%
\newpage
%%%%%%%%%%%%%%%%%%%%%%%%%%%%%%%%%
\section{Exact differential equations: solving}

\subsection*{Resources}
\begin{itemize}
    \item Videos: Exact equations examples 1,2,3 starting from \url{https://www.khanacademy.org/math/differential-equations/first-order-differential-equations/exact-equations/v/exact-equations-example-1}
    \item Text: \url{http://tutorial.math.lamar.edu/Classes/DE/Exact.aspx}
\end{itemize}

\subsection*{Challenge}
In challenge \ref{sec:edeid} you should have identified 4 exact differential equations. Considering each of the 4 EDE's in order, try to solve the EDE's applying the following conditions:

\subsubsection{1st EDE}
Do not try to solve this one.

\subsubsection{2nd EDE}
Use the condition $y(\pi/4)=\pi/4$ to find an explicit solution for the equation and then evaluate $y$ at $x=\pi$.

\subsubsection{3rd EDE}
Use the condition $y(1)=3$ to find an explicit solution for the equation and then evaluate $y$ at $x=4$.

\subsubsection{4th EDE}
Use the condition $y(1)=2$ to find an explicit solution for the equation and then evaluate $y$ at $x=1$.


\subsection*{Solution}

\subsubsection{2nd EDE}
\six{}

\hash{m}{af87e2}

\subsubsection{3rd EDE}
\six{}

\hash{aa}{d01c3d}

\subsubsection{4th EDE}
\six{}

\hash{bb}{5e1074}




%%%%%%%%%%%%%%%%%%%%%%%%%%%%%%%%%
\newpage
%%%%%%%%%%%%%%%%%%%%%%%%%%%%%%%%%
\section{Exact differential equations: a useful integration method}

\subsection*{Challenge}
Obtain an expression for $g(x)$ in terms of $f(x)$ in the following integral:

\begin{equation}
    \int \frac{f'(x)}{f(x)} dx = g(x)
\end{equation}

ie, you should be able to re-write $g(x)$ in terms of a simple (non-integral) function of $f(x)$, in the form $g(x) = \cdots$.

\subsection*{Solution}
You can check your answer by putting a function of $x$ into $f(x)$.




%%%%%%%%%%%%%%%%%%%%%%%%%%%%%%%%%
\newpage
%%%%%%%%%%%%%%%%%%%%%%%%%%%%%%%%%
\section{Exact differential equations: integrating factors}
\label{sec:edeif}

\subsection*{Resources}
\begin{itemize}
    \item Videos: Integrating factors 1,2 starting from \url{https://www.khanacademy.org/math/differential-equations/first-order-differential-equations/exact-equations/v/integrating-factors-1}
\end{itemize}

\section*{Comment}
Note that in the videos, Sal Khan does an example considering an integrating factor of $\mu(x)$, but in some cases $\mu(y)$ leads to a solution more easily. You may need to try both to determine an answer.

\subsection*{Challenge}
Solve the exact differential equations below using integrating factors.

1. Solve the equation below using an integrating factor. Place the solution in the form $f(x,y) = C$, then calculate the value of $C$ when substituting $x=2$ and $y=1$ into the equation. Do not try to solve the equation to get it in the form $y(x)=\cdots$.

\begin{equation}
    \label{eq:edeif1}
    y dx + (2 x y - e^{-2 y}) dy = 0
\end{equation}

2. Calculate the integrating factor for the following equation. To check your answer, substitute $x=1$ or $y=1$ into any final expression, assuming an integration constant of zero.

\begin{equation}
    \label{eq:edeif2}
    y(3x-y) dx + x(x-y)dy = 0
\end{equation}

3. Show that $1/(x^y+y^2)$ is an integrating factor for the equation
\begin{equation}
    x dx + y dy + 4 y^3 (x^2 + y^2)dy = 0
\end{equation}

\subsection*{Solution}
Challenge related to equation \ref{eq:edeif1}: \hash{cc}{bb15d6}

Challenge related to equation \ref{eq:edeif2}: \hash{dd}{6a8742}




%%%%%%%%%%%%%%%%%%%%%%%%%%%%%%%%%
\newpage
%%%%%%%%%%%%%%%%%%%%%%%%%%%%%%%%%
\section{Exact differential equations: integrating factor derivation}
\label{sec:intfacderiv}

\subsection*{Challenge}
1. Starting from the equation

\begin{equation}
    \mu(x,y) M(x,y) dx + \mu(x,y) N(x,y) dy = 0
\end{equation}

show that if the integrating factor $\mu$ is only a function of $x$, then

\begin{equation}
    \label{eq:intfacmux}
    \mu_x = \mu \left ( \frac{M_y-N_x}{N} \right )
\end{equation}

2. Do the same, assuming that $\mu$ is only a function of $y$.

% NT do something for linear equations mu=e^\int(a)




%%%%%%%%%%%%%%%%%%%%%%%%%%%%%%%%%
\newpage
%%%%%%%%%%%%%%%%%%%%%%%%%%%%%%%%%
\section{Exact differential equations: integrating factor calculation}
\label{sec:edeifcalc}

\section*{Comment}
Without proof, we can use equation \ref{eq:intfacmux} to gain information about the existance of an integration factor. If $\left ( \frac{M_y-N_x}{N} \right )$ is a function of $x$ only, then we know that the integration factor is only a function of $x$, and it can be solved for by integration of equation \ref{eq:intfacmux}. The same can be said for $\mu(y)$ that you derived an expression for in challenge \ref{sec:intfacderiv}.

\subsection*{Challenge}
Use equations from section \ref{sec:intfacderiv} and information provided in the comment here to determine the integrating factor for

\begin{equation}
    \label{eq:edeifcalc1}
    e^x dx + (e^x Cot(y) + 2y Csc(y)) dy = 0
\end{equation}

and

\begin{equation}
    \label{eq:edeifcalc2}
    (x-y^2) dx + 2xy dy = 0
\end{equation}

To check your answer, for both cases substitute $x=\pi$ or $y=\pi$ into the integrating factor, and assume an integration constant of 1.

\subsection*{Solution}


Equation \ref{eq:edeifcalc1}: \hash{ee}{51a0ae}

Equation \ref{eq:edeifcalc2}: \hash{ff}{a56bce}




%%%%%%%%%%%%%%%%%%%%%%%%%%%%%%%%%
\newpage
%%%%%%%%%%%%%%%%%%%%%%%%%%%%%%%%%
\section{Summary of 1st-order differential equations}

\subsection*{Challenge}
1. Create a flowchart describing how you will approach solving a general 1st-order differential equation.

% Linear, Separable, Exact (with and without integration factors)
2. Solve the following 1st-order differential equations:

\begin{equation}
    \label{eq:1odegen1}
    y' - 4y = 8x + 3
\end{equation}
evaluated at $x=1$.

\begin{equation}
    \label{eq:1odegen2}
    4yy' = 8x + 3
\end{equation}
assuming an integration constant of zero and evaluating the final equation at $x=2$.

\begin{equation}
    \label{eq:1odegen3}
    y' + 4y = e^{-8x}
\end{equation}
assuming an integration constant of zero and evaluating the final equation at $x=1/8$.

\subsection*{Solution}
Equation \ref{eq:1odegen1}: \hash{qq}{a43ab2}

Equation \ref{eq:1odegen2}: \hash{rr}{990bfa}

Equation \ref{eq:1odegen3}: \hash{ss}{91989d}

\fi



% Remaining Khan 1st-order equations
% Benouli and substitution

% About 4.5 weeks on 1st-order and 4.5 weeks on 2nd order?
% Khan covers 1st-order well but 2nd-order not so much.
% Follow Khan for 1st-order, then do 2nd order and fill in holes with Paul's notes - this may cause issues if terminology is different.
% More advanced topics are in Paul's notes, but first check how much time is remaining.

% Do I need to add more challenges for simple 1st order linear equations?

\chapter{2nd-order differential equations}
\section{Hooke's law}

\subsection*{Resources}

\includegraphics[scale=0.5]{hook.png}\\
\emph{(\href{http://hyperphysics.phy-astr.gsu.edu/hbase/imgmec/hook.gif}{Image} from HyperPhysics by Rod Nave, Georgia State University)}

\subsection*{Challenge}
2nd-order differential equations deal with oscillations.

Considering Hooke's law, what are $A$ and $C$ in the following equation?
\begin{equation}
    A x'' + C x = 0
\end{equation}

To check your answer, substitute a mass of \SI{2}{kg} and spring-constant of \SI{3}{kg/s^2} as appropriate.

\subsection*{Solution}
Enter only numerical values without units such as kg.

A: \hash{gg}{4e5fe6}

C: \hash{hh}{6a7015}




%%%%%%%%%%%%%%%%%%%%%%%%%%%%%%%%%
\newpage
%%%%%%%%%%%%%%%%%%%%%%%%%%%%%%%%%
\section{Exponentials and trigonometry}

\subsection*{Resources}
\begin{itemize}
    \item Text: \url{https://www.phy.duke.edu/~rgb/Class/phy51/phy51/node15.html}
\end{itemize}

\subsection*{Challenge}
Write $sin(x)$ and $cos(x)$ in exponential form.

\subsection*{Solution}

Check your answer with someone if you are unsure.


\timebox




%%%%%%%%%%%%%%%%%%%%%%%%%%%%%%%%%
\newpage
%%%%%%%%%%%%%%%%%%%%%%%%%%%%%%%%%
\section{Characteristic equation: understanding}

\subsection*{Resources}
\begin{itemize}
    \item Book (\url{http://tutorial.math.lamar.edu/getfile.aspx?file=B,1,N}) from page 111.
\end{itemize}

\subsection*{Comment}
A homogeneous (ie, equal to zero) second-order differential equation typically takes the form:

\begin{equation}
    A \frac{d^2y}{dt^2} + B \frac{dy}{dt} + C y = 0
\end{equation}

The first (A) term describes acceleration, while the third (C) term is the force-constant term (something like the ``stiffness'' of the spring). The second (B) term could describe a frictional force that is proportional to the velocity ($dy/dt$). Due to its relation with oscillation (and by extension, sines and cosines which can be expressed in terms of exponentials) we can typically assume an exponential-form solution to the differential equation.

\subsection*{Challenge}
Show that, assuming that all solutions to a 2nd-order differential equation of the form above will have solutions $y(t)=e^{rt}$, the value of $r$ can in principle be determined by solving the following a quadratic equation of the form
\begin{equation}
    A r^2 + Br + C = 0
\end{equation}

\subsection*{Solution}
If you are unsure of your derivation, please ask someone.

\timebox




%%%%%%%%%%%%%%%%%%%%%%%%%%%%%%%%%
\newpage
%%%%%%%%%%%%%%%%%%%%%%%%%%%%%%%%%
\section{Characteristic equation: roots}

\subsection*{Resources}
\begin{itemize}
    \item Book (\url{http://tutorial.math.lamar.edu/getfile.aspx?file=B,1,N}) from page 111.
\end{itemize}

\subsection*{Challenge}
Sum the points of the differential equations that have characteristic equations with
\begin{itemize}
    \item Real, distinct roots
    \item Complex roots
    \item Equal roots
\end{itemize}

1 point: $\displaystyle -3 y'' - 5 y' + 2 y = 0$ % C

2 points: $\displaystyle 3 y'' - 4 y' + 3 y = 0$ % E

4 points: $\displaystyle 3 y'' - 6 y' + 3 y = 0$ % B

8 points: $\displaystyle 3 y'' - 5 y' + 2 y = 0$ % F

16 points: $\displaystyle 3 y'' - 5 y' + 4 y = 0$ % D

32 points: $\displaystyle 3 y'' + 5 y' + 2 y = 0$ % A

\subsection*{Solution}

\begin{itemize}
    \item Real, distinct roots: \hash{ii}{064a6e}
    \item Complex roots: \hash{jj}{5cdb6c}
    \item Equal roots: \hash{kk}{70cd8f}
\end{itemize}

\timebox




%%%%%%%%%%%%%%%%%%%%%%%%%%%%%%%%%
\newpage
%%%%%%%%%%%%%%%%%%%%%%%%%%%%%%%%%
\section{Characteristic equation: real roots with positive B}

\subsection*{Resources}
\begin{itemize}
    \item Book (\url{http://tutorial.math.lamar.edu/getfile.aspx?file=B,1,N}) from page 116.
\end{itemize}

\subsection*{Challenge}
Solve the following 2nd-order differential equation that has real roots:

\begin{equation}
    \label{eq:ccrrpb}
    y'' + 3 y' + 2 y = 0
\end{equation}

with initial conditions $y(0)=5$ and $y'(0)=-8$.

To check your answer, substitute $t=1$ into the final expression.


\subsection*{Solution}
1.14 %\hash{mm}{9b9be5}




%%%%%%%%%%%%%%%%%%%%%%%%%%%%%%%%%
\newpage
%%%%%%%%%%%%%%%%%%%%%%%%%%%%%%%%%
\section{Characteristic equation: real roots with negative B}

\subsection*{Resources}
\begin{itemize}
    \item Book (\url{http://tutorial.math.lamar.edu/getfile.aspx?file=B,1,N}) from page 116.
\end{itemize}

\subsection*{Challenge}
Solve the following 2nd-order differential equation that has real roots. 

\begin{equation}
    y'' - 3 y' + 2 y = 0
\end{equation}

with initial conditions $y(0)=5$ and $y'(0)=-8$. Substitute $t=1$ into the final expression to check your answer.

Note that this equation is the same as equation \ref{eq:ccrrpb}, but simply the dampening (friction) term B has been changed from positive to negative.


\subsection*{Solution}
-47.13 %\hash{nn}{473835}

\timebox




%%%%%%%%%%%%%%%%%%%%%%%%%%%%%%%%%
\newpage
%%%%%%%%%%%%%%%%%%%%%%%%%%%%%%%%%
\section{Characteristic equation: B in equations with real roots}

\subsection*{Challenge}

\emph{(Note that there are two parts to this challenge.)}

1. Considering real root, sum the points of the following true statements:

Considering the equation

\begin{equation}
    A y'' + B y' + C y = 0
\end{equation}

1 point: Positive damping (positive B) leads to solutions with exponentials with positive exponents.

2 points: Positive damping (positive B) leads to solutions with exponentials with negative exponents.

4 points: Negative damping (negative B) leads to solutions with exponentials with positive exponents.

8 points: Negative damping (negative B) leads to solutions with exponentials with negative exponents.

16 points: Exponentials with positive exponents (eg, $e^{t}$) lead to exponential growth (instability).

32 points: Exponentials with negative exponents (eg, $e^{-t}$) lead to exponential growth (instability).

64 points: Exponentials with positive exponents (eg, $e^{t}$) lead to a damped signal (stability).

128 points: Exponentials with negative exponents (eg, $e^{-t}$) lead to damped signal (stability).

\vspace{2em}

2. Write a sentence summarising your understanding of the significance of having a positive or negative coefficient of $B$ when the roots are real.


\subsection*{Solution}
\hash{oo}{fa6adf}

\timebox




%%%%%%%%%%%%%%%%%%%%%%%%%%%%%%%%%
\newpage
%%%%%%%%%%%%%%%%%%%%%%%%%%%%%%%%%
\section{Characteristic equation: equal roots}

\subsection*{Resources}
\begin{itemize}
    \item Book (\url{http://tutorial.math.lamar.edu/getfile.aspx?file=B,1,N}) from page 125.
\end{itemize}

\subsection*{Comment}
It is not necessary to follow the full derivation in the suggested resource.

\subsection*{Challenge}
Solve the equation
\begin{equation}
    y'' - 2y' + y = 0
\end{equation}

To check your solution, substitute $t=1$ into the equation and assume $c_1 = c_2 = 1$.

\subsection*{Solution}
5.44 %\hash{pp}{ff7ca2}

\timebox




%%%%%%%%%%%%%%%%%%%%%%%%%%%%%%%%%
\newpage
%%%%%%%%%%%%%%%%%%%%%%%%%%%%%%%%%
\section{Characteristic equation: complex roots with B=0}

\subsection*{Resources}
\begin{itemize}
    \item Book (\url{http://tutorial.math.lamar.edu/getfile.aspx?file=B,1,N}) from page 120.
\end{itemize}

\subsection*{Challenge}
1. Assuming there is no damping term (ie, $B=0$) show that the roots for the differential equation
\begin{equation}
    A y'' + Cy = 0
\end{equation}
are $\pm i \sqrt{C/A}$.

2. Solve the following ODE:
\begin{equation}
    \label{eq:cecr}
    y'' + 4 \pi^2 y = 0
\end{equation}

To check your answer, assume integration constants of 1 and calculate $y(\pi/2)$.

\subsection*{Solution}
Solution to part 2: -1.33 %\hash{qq}{7eb2c9}

\timebox




%%%%%%%%%%%%%%%%%%%%%%%%%%%%%%%%%
\newpage
%%%%%%%%%%%%%%%%%%%%%%%%%%%%%%%%%
\section{Characteristic equation: complex roots with positive B}

\subsection*{Resources}
\begin{itemize}
    \item Book (\url{http://tutorial.math.lamar.edu/getfile.aspx?file=B,1,N}) from page 120.
\end{itemize}

\subsection*{Challenge}
Solve the following ODE:
\begin{equation}
    y'' + y' + y = 0
\end{equation}

To check your answer, assume integration constants of 1 and calculate $y(\pi/2)$.

\subsection*{Solution}
-0.61 %\hash{rr}{1d0cb5}

\timebox




%%%%%%%%%%%%%%%%%%%%%%%%%%%%%%%%%
\newpage
%%%%%%%%%%%%%%%%%%%%%%%%%%%%%%%%%
\section{Characteristic equation: complex roots with negative B}

\subsection*{Resources}
\begin{itemize}
    \item Book (\url{http://tutorial.math.lamar.edu/getfile.aspx?file=B,1,N}) from page 120.
\end{itemize}

\subsection*{Challenge}
Solve the following ODE:
\begin{equation}
    y'' - y' + y = 0
\end{equation}

To check your answer, assume integration constants of 1 and calculate $y(\pi/2)$.

\subsection*{Solution}
-2.92 %\hash{ss}{caf35b}

\timebox




%%%%%%%%%%%%%%%%%%%%%%%%%%%%%%%%%
\newpage
%%%%%%%%%%%%%%%%%%%%%%%%%%%%%%%%%
\section{Damping}
\label{sec:damping}

\subsection*{Resources}
\begin{itemize}
    \item Wikipedia: \url{https://en.wikipedia.org/wiki/Damping}
\end{itemize}

\subsection*{Challenge}
Of the 6 functions shown in the graph, place the 3 that correspond to over-damped, critically damped and under-damped in the order mentioned in this sentence.

\includegraphics[scale=0.75]{damping.png}

\subsection*{Solution}
(eg, ``abc'')

\hash{tt}{060b2a}

\timebox



%%%%%%%%%%%%%%%%%%%%%%%%%%%%%%%%%
\newpage
%%%%%%%%%%%%%%%%%%%%%%%%%%%%%%%%%
\section{Damping and 2nd-order differential equations}

\subsection*{Challenge}
1. The 6 functions shown in the graph in challenge \ref{sec:damping} may represent solutions of a 2nd-order differential equation. Place the solutions A-F in the order shown below. Note that one of the descriptions below is impossible, and you should ignore that one.

I. Solution of a 2nd-order differential equation with real roots and positive B.

II. Solution of a 2nd-order differential equation with real roots and negative B.

III. Solution of a 2nd-order differential equation with real roots and B=0.

IV. Solution of a 2nd-order differential equation with equal roots.

V. Solution of a 2nd-order differential equation with complex roots and B=0.

VI. Solution of a 2nd-order differential equation with complex roots and positive B.

VII. Solution of a 2nd-order differential equation with complex roots and negative B.

\vspace{1em}
2. Write one sentence stating why one of the above solutions is impossible.

\subsection*{Solution}
(eg, ``abcdef'')

\hash{uu}{a96870}

\timebox




%%%%%%%%%%%%%%%%%%%%%%%%%%%%%%%%%
\newpage
%%%%%%%%%%%%%%%%%%%%%%%%%%%%%%%%%
\section{The Wronskian}

\subsection*{Resources}
\begin{itemize}
    \item Book (\url{http://tutorial.math.lamar.edu/getfile.aspx?file=B,1,N}) from page 133.
\end{itemize}

\subsection*{Challenge}

\emph{Please write the following answers clearly and in a manner that can be easily shared with others in the class.}

1. What is meant by a ``fundamental set of solutions''?

2. Why is the final solution for real and complex roots always a sum of two terms?

3. Why do the two terms need to be linearly independent? What would happen if they were not linearly independent?

4. What is the ``Wronskian'', and what is the formula for its calculation?

5. How can the Wronskian be used to determine linear independence?

\subsection*{Solution}
Please read at least 1 other peer's solution and discuss any differences. The teacher will also help check your understanding.

\timebox




%%%%%%%%%%%%%%%%%%%%%%%%%%%%%%%%
\newpage
%%%%%%%%%%%%%%%%%%%%%%%%%%%%%%%%
\section{Characteristic equation: exercises}

\emph{(Note that if you encounter a square-root during your calculations such as $\sqrt{7}$, it is best to work with $\sqrt{7}$ rather than $2.65$ in order to maintain accuracy until the final step where you need to evaluate it. If the equation becomes too messy (eg $e^{(\sqrt{7}-1)/\sqrt{3}}$) you can always substitute $m=(\sqrt{7}-1)/\sqrt{3}$, etc, to make things clearer.)}

\subsection*{Challenge}
1. Determine $y(1)$ for the equation

\begin{equation}
    2 y''+8y'+y=0    
\end{equation}
given the initial conditions $y(0)=4$ and $y'(0)=3$.

2. Determine $y(0.2)$ for the equation

\begin{equation}
    2y''+4y'+2y=0
\end{equation}
given the initial conditions $y(0)=4$ and $y'(0)=2$.

3. Determine $y(0.1)$ for the equation

\begin{equation}
    4y''+3y'+y=0
\end{equation}
given the initial conditions $y(0)=6$ and $y'(0)=2$.


\subsection*{Solution}
1. 4.32 %\hash{vv}{f01192}

2. 4.26 %\hash{ww}{6f5d64}

3. 6.19 %\hash{xx}{c74f58}

\timebox




%%%%%%%%%%%%%%%%%%%%%%%%%%%%%%%%
\newpage
%%%%%%%%%%%%%%%%%%%%%%%%%%%%%%%%
\section{(C19,C20,C21,C22) Non-homogeneous equations: Method of undetermined coefficients}

\subsection*{Resources}
\begin{itemize}
    \item Video: All 4 Khan Academy videos starting at \url{https://www.khanacademy.org/math/differential-equations/second-order-differential-equations/undetermined-coefficients/v/undetermined-coefficients-1}
\end{itemize}

\subsection*{Comment}
The 2nd-order equations we were considering until now were homogeneous equations (ie, the RHS was zero). We can now build upon this to expand our ability to solve non-homogeneous equations (ie, where the RHS of the equation is non-zero).

The Khan Academy videos give an excellent initial introduction to the subject, and so please do take the time to view and take notes about all four videos in the series.

In the 4th video Mr Kahn describes about how it is possible to add solutions if there are multiple terms on the right. This occasionally causes confusion. Consider for example:

\begin{equation}
    y'' - 3y' - 4y = 2 \sin x
\end{equation}

This corresponds to the particular solution

\begin{equation}
    y_p = A \sin x + B \cos x
\end{equation}

A common point of confusion is about what to do in the case of something like
\begin{equation}
    \label{eq:particularconfusion}
    y'' - 3y' - 4y = 2 \sin x + 2 \cos x
\end{equation}

Should you just write $y_p = (A \sin x + B \cos x) + (C \sin x + D \cos x)$? After all, you have two terms in equation \ref{eq:particularconfusion} (ie, $2 \sin x$ and $2 \cos x$). You can note however that $A \sin x + C \sin x$ simplifies to $E \sin x$ where $E$ is just another constant (in this case $A+B$) so in the end you will be left with $y_p = E \sin x + F \cos x$. So while it may be clearer to explicitly calculate coefficients for every term on the RHS, in many cases the terms will simplify.


\subsection*{Challenge}
Find the general solution of the following non-homogeneous differential equations:

1. (C19) $y'' + 4y = 8$\\
2. (C20) $y'' + 4y = 8t^2 - 20t + 8$\\
3. (C21) $y'' + 4y = 5 \sin 3t - 5 \cos 3t$\\
4. (C22) $y'' + 4y = 24 e^{-2t}$

(please just rate the challenges on challenge-hub after you have determined the answer for each one)

\subsection*{Solution}
The solutions are contained in the list on the next page in no particular order. Your answers should match one of the solutions given. Please try to not look at the solutions before completing the questions, since this will facilitate deep understanding and reproduce a real-life/exam environment.
\newpage
$y = C_1 \cos 2t + C_2 \sin 2t + 3e^{-2t}$\\ %4
$y = C_1 \cos 2t + C_2 \sin 2t + 8e^{-2t}$\\
$y = C_1 \cos 2t + C_2 \sin 2t + 2t^2 - 5t + 1$\\ %2
$y = C_1 \cos 2t + C_2 \sin 2t + 3t^2 + t + 3$\\
$y = C_1 \cos 2t + C_2 \sin 2t + \cos 3t - \sin 3t$\\ %3
$y = C_1 \cos 2t + C_2 \sin 2t + 2$\\ %1
$y = C_1 \cos 2t + C_2 \sin 2t + 5$\\



%%%%%%%%%%%%%%%%%%%%%%%%%%%%%%%%
\newpage
%%%%%%%%%%%%%%%%%%%%%%%%%%%%%%%%
\section{(C23-28) Method of undetermined coefficients II}

\subsection*{Comment}
The following pages go into more detail than the videos, considering a greater range of cases. You may note that here the particular solution is denoted by $Y$ while Sal Khan denoted it as $y_p$ in the videos.

\emph{The following notes were developed by Zachary S. Tseng at Pennsylvania State University, USA (\url{http://www.math.psu.edu/tseng/}). Included here with kind permission.}

\includepdf[pages=-,pagecommand={},width=\textwidth,nup=1x1,frame=true]{External/undetermined.pdf}

\subsection*{Challenge}
The following challenges expand the range of problems to give you practise in a range of situations.

1. (C23) $y'' + 4y = 8 \cos 2t$\\
2. (C24) $y'' + 2y' = 2 te^{-t}$\\
3. (C25) $y'' + 2y' = 6 e^{-2t}$\\
4. (C26) $y'' + 2y' = 12 t^2$\\
5. (C27) $y'' - 6y' - 7y = 13 \cos 2t + 34 \sin 2t$\\
6. (C28) $y'' - 6y' - 7y = 8e^{-t} - 7t - 6$

\subsection*{Solution}
Assuming that the constants you find in your solution are all equal to 1, check your answer by calculating $y(t=0.4)$ in each case. To check your answer, please subsume all constants on any term into the constant that you set to 1. For example, instead of $y(t) = -2 C_1 e^{-t} + e^{-t}$, write $y(t) = C_1 e^{-t}$ where the two $e^{-t}$ terms have been combined and the $-2$ has been subsumed into the constant $C_1$, and then set $C_1 = 1$ to check the answer.




%%%%%%%%%%%%%%%%%%%%%%%%%%%%%%%%
\newpage
%%%%%%%%%%%%%%%%%%%%%%%%%%%%%%%%
\section{(C29) Method of undetermined coefficients: Determining the ODE I}

\subsection*{Comment}
This challenge gives you useful practise of going the other way; determining a differential equation that describes a given solution.

\subsection*{Challenge}
Determine the 2nd-order linear differential equation which has the general solution
\begin{equation}
    y = C_1 \cos 4t + C_2 \sin 4t - e^t \sin 2t
\end{equation}

\subsection*{Solution}
You will end up with differential terms on the left side and a function of $t$ on the right side.
Please compare your answer with your partner in class.
To check your answer with challenge-hub, evaluate the right side of your equation by substituting the value $t=0.4$.




%%%%%%%%%%%%%%%%%%%%%%%%%%%%%%%%
\newpage
%%%%%%%%%%%%%%%%%%%%%%%%%%%%%%%%
\section{(C30) Method of undetermined coefficients: Determining the ODE II}

\subsection*{Comment}
This challenge gives you useful practise of going the other way; determining a differential equation that describes a given solution.

\subsection*{Challenge}
Determine the 2nd-order linear differential equation which has the general solution
\begin{equation}
    y = C_1 e^{-2t} + C_2 t e^{-2t} + t^3 - 3t
\end{equation}

\subsection*{Solution}
You will end up with differential terms on the left side and a function of $t$ on the right side.
Please compare your answer with your partner in class.
To check your answer with challenge-hub, evaluate the right side of your equation by substituting the value $t=0.4$.


\chapter{Laplace transformation}
\section{Your first Laplace Transform calculations}

\subsection*{Resources}
\begin{itemize}
    \item Videos: The \textbf{four} Khan-academy videos starting at \url{https://www.khanacademy.org/math/differential-equations/laplace-transform/laplace-transform-tutorial/v/laplace-transform-1} % 8m+7:30+10+9 = 35:30
\end{itemize}

\subsection*{Comment}
The Laplace Transform is a powerful technique that has many uses beyond solving ODE's. It can however appear a bit abstract at first. Becoming comfortable with controlling and manipulating the transform will help provide confidence when using it to solve ODE's. The four videos in the resources above provide an excellent starting point for getting you comfortable with this powerful technique.

\subsection*{Challenge}
1. Calculate $\lap{1}$

2. Calculate $\lap{at}$

3. Calculate $\lap{Cos(at)}$

\subsection*{Solution}
To check your answer, substitute $s=1$ and $a=2$ into your final solution.

1. 1

2. 2

3. $\frac{1}{5}$




%%%%%%%%%%%%%%%%%%%%%%%%%%%%%%%%
\newpage
%%%%%%%%%%%%%%%%%%%%%%%%%%%%%%%%
\section{Laplace transform of a 3rd derivative}

\subsection*{Resources}
\begin{itemize} % Total video time: 20m
    \item Video I: \url{https://www.khanacademy.org/math/differential-equations/laplace-transform/properties-of-laplace-transform/v/laplace-transform-5} % 11:30
    \item Video II: \url{https://www.khanacademy.org/math/differential-equations/laplace-transform/properties-of-laplace-transform/v/laplace-transform-6} % 9:30
\end{itemize}

\subsection*{Challenge}
1. Calculate $\displaystyle \frac{d^3}{dt^3} \left( t e^{a t} \right)$

2. Given
\begin{equation}
    \lap{t e^{a t}} = \frac{1}{(a-s)^2}
\end{equation}
determine $\lap{3a^2 e^{at} + a^3te^{at}}$

\subsection*{Solution}
To check your answer, substitute $s=1$ and $a=2$ into your final solution.

-4




%%%%%%%%%%%%%%%%%%%%%%%%%%%%%%%%
\newpage
%%%%%%%%%%%%%%%%%%%%%%%%%%%%%%%%
\section{Shifting a transform}

\subsection*{Resources}
\begin{itemize}
    \item Video: \url{https://www.khanacademy.org/math/differential-equations/laplace-transform/properties-of-laplace-transform/v/more-laplace-transform-tools} %11m
\end{itemize}

\subsection*{Challenge}
Given
\begin{equation}
    \lap{Cosh(at)} = \frac{s}{s^2-a^2}
\end{equation}

1. What is $\lap{e^{3t} Cosh(5t)}$?

2. What is $f(t)$ in the equation $\lap{f(t)} = \frac{s-4}{(s-4)^2-100}$?

\subsection*{Solution}
To check your answer, substitute $s=2$ and $t=2$ as appropriate:

1. $0.0417$

2. $7.23 \times 10^{11}$




%%%%%%%%%%%%%%%%%%%%%%%%%%%%%%%%
\newpage
%%%%%%%%%%%%%%%%%%%%%%%%%%%%%%%%
\section{L'H\^opital's rule}

\subsection*{Resources}
\begin{itemize}
    \item Wikipedia: \url{https://en.wikipedia.org/wiki/L\%27H\%C3\%B4pital\%27s_rule}
\end{itemize}

\subsection*{Challenge}
1. Use L'H\^opital's rule to determine the limit of
\begin{equation}
    t e^{-st}
\end{equation}
as $t \rightarrow 0$.

2. Considering the case of
\begin{equation}
    \frac{t^n}{e^{st}}
\end{equation}
if we apply L'H\^opital's rule $n$ times with respect to $t$, what is the power of $t$ in the numerator? Note that $e^{st}$ is always constant, so by repeated differentiation we can apply L'H\^opital's rule even for $t^n$.

\subsection*{Solution}
1. \hash{ww}{76c8d4}

2. \hash{xx}{1592d7}




%%%%%%%%%%%%%%%%%%%%%%%%%%%%%%%%
\newpage
%%%%%%%%%%%%%%%%%%%%%%%%%%%%%%%%
\section{Laplace Transformation of the unit step function}

\subsection*{Resources}
\begin{itemize}
    \item Video: \url{https://www.khanacademy.org/math/differential-equations/laplace-transform/properties-of-laplace-transform/v/laplace-transform-of-the-unit-step-function} % 24m
\end{itemize}

\subsection*{Challenge}
Considering $U_c$ as the unit step-function at $c$, calculate the following Laplace transformations:

1. $\displaystyle \lap{U_0}$

2. $\displaystyle \lap{U_c}$

3. A 1-second pulse function starting at time $t=1$ with value $f(y)=1$ as shown in the graph below:
\includegraphics[scale=0.5]{pulse.png}

4. $\displaystyle \lap{U_\pi(t) cos(t-\pi)}$

\subsection*{Solution}
To check your answers, substitute $c=1$ and $s=2$ as appropriate.

1. \hash{yy}{39574c}

2. 0.0677

3. 0.0585

4. $7.470 \times 10^{-4}$




%%%%%%%%%%%%%%%%%%%%%%%%%%%%%%%%
\newpage
%%%%%%%%%%%%%%%%%%%%%%%%%%%%%%%%
\section{Inverse Laplace Transform}

\subsection*{Resources}
\begin{itemize}
    \item Video: \url{https://www.khanacademy.org/math/differential-equations/laplace-transform/properties-of-laplace-transform/v/inverse-laplace-examples} % 19m
\end{itemize}

\subsection*{Comment}
Being able to reversing the Laplace transform is a crucial skill required for applying it to solving ODE's. It can be a little confusing at first however, so I recommend to take your time to understand the essential steps involved thoroughly, as this will then give you greater confidence when you come to apply this to solving ODE's. To this end, the video listed in the resource is a fantastic introduction to this.

\subsection*{Challenge}
Determine the function $f(t)$ by finding the inverse of the following Laplace transforms:

1. $\displaystyle F(s)=\frac{1}{(s-1)^2}$

2. $\displaystyle F(s)=\frac{1-s}{s^2}$

3. $\displaystyle F(s)=\frac{2 e^{-2s}}{s^2-2s+2}$

4. $\displaystyle F(s)=\frac{6}{(2+s)^4}$

5. $\displaystyle F(s)=\frac{120+6s^3}{s^6}$

6. $\displaystyle F(s)=\frac{e^{12-3s}}{s-4}$


\subsection*{Solution}
To check your answers, substitute $t=2$ into your final answer. If there is a unit-step in your solution, precede your numerical answer with ``u(c)'' where ``c'' is the position of the unit step. So for example, an answer of $U_5 t^2$ would be entered as ``u(5.00)4.00'' (all numbers to two decimal places). An answer without a unit-step would just be entered to two decimal places (eg, ``4.00'' in the previous example).

1. Hash = 5cacdb\ldots

2. Hash = 41cf26\ldots

3. Hash = 45c11e\ldots

4. Hash = 9ffc7a\ldots

5. Hash = 766fd0\ldots

6. Hash = e60079\ldots




%%%%%%%%%%%%%%%%%%%%%%%%%%%%%%%%
\newpage
%%%%%%%%%%%%%%%%%%%%%%%%%%%%%%%%
\section{The Dirac delta function and its Laplace transform}

\subsection*{Resources}
\begin{itemize}
    \item Video I: \url{https://www.khanacademy.org/math/differential-equations/laplace-transform/properties-of-laplace-transform/v/dirac-delta-function}
    \item Video II: \url{https://www.khanacademy.org/math/differential-equations/laplace-transform/properties-of-laplace-transform/v/laplace-transform-of-the-dirac-delta-function}
\end{itemize}

\subsection*{Challenge}
Calculate the following Laplace transforms (treat $c$ as a positive constant):

1. $\displaystyle \lap{\delta(t)}$

2. $\displaystyle \lap{\delta(t-c)}$

3. $\displaystyle \lap{\delta(t-2) Cos(4 t)}$

4. $\displaystyle \lap{\delta(t) (t^2+10)}$

\subsection*{Solution}
To check your solution, set $s=1$, $c=2$ and $t=1$ as appropriate to check your answers.

1. \hash{zz}{ffef92}

2. \hash{aaa}{826784}

3. \hash{bbb}{f44448}

4. \hash{ccc}{4ca484}




%%%%%%%%%%%%%%%%%%%%%%%%%%%%%%%%
\newpage
%%%%%%%%%%%%%%%%%%%%%%%%%%%%%%%%
\section{The Dirac delta function and its inverse Laplace transform}

\subsection*{Challenge}
Calculate the following Laplace transform:

$\displaystyle \delta(t-2) Sin(2t)$

Calculate the following inverse Laplace transforms:

1. $\displaystyle e^{-2s} Sin(2)$

2. $\displaystyle e^{-2s} Sin(4)$

\subsection*{Solution}
To check your answer, substitute $t=1$ into the final expression and evaluate the part inside and outside of the Dirac delta function separately. So for example, if your answer is $\delta(t-2) (t^2+1)$, the expression inside the delta-function is $t-2$ and will evaluate to $-1.00$ while the expression outside of the delta-function is $t^2+1$ and will evaluate to $2.00$.

1. Inside delta function: \hash{ddd}{7cec9e}; Outside delta function: \hash{eee}{8147e6}

2. Inside delta function: \hash{fff}{033c55}; Outside delta function: \hash{ggg}{b1643a}




%%%%%%%%%%%%%%%%%%%%%%%%%%%%%%%%
\newpage
%%%%%%%%%%%%%%%%%%%%%%%%%%%%%%%%
\section{A forced spring}

\begin{itemize}
    \item The \textbf{four} videos starting at \url{https://www.khanacademy.org/math/differential-equations/laplace-transform/laplace-transform-to-solve-differential-equation/v/laplace-transform-to-solve-an-equation}
    \item A useful table of Laplace transforms: \url{http://tutorial.math.lamar.edu/pdf/Laplace_Table.pdf}
\end{itemize}

\section*{Comment}
Here you finally get the opportunity to practise solving ODE's using the powerful method of Laplace transformations. Please takes notes from all four videos listed in the resources section; they provide very useful examples of how to use this method, including related algebraic techniques that are commonly required to solve such challenges.

\subsection*{Challenge}
The spring equation you encountered in challenge \ref{sec:hooke} introduced you to the concept of oscillation of a mass on a spring. There, the equation to determine the displacement of the spring $y$ from its equilibrium position was $y''+y=0$, which yields a solution $y=C_1 Cos(t) + C_2 Sin(t)$. This is free oscillation without external damping or driving, and it will oscillate according to the cosine and sine sum for all time ($t$). It is also possible to add a forcing term to the equation by making it non-homogeneous, such as in the form

\begin{equation}
    y'' + 4y = 2 Cos(3t)
\end{equation}

Here the forcing varies with time $t$ in the form of a cosine wave.

Use the Laplace transform method to solve the ODE in the above equation given a starting displacement of zero and an initial velocity of zero. You may use the table of Laplace transforms in the resources to help you.

\subsection*{Solution}
Substitute $t=1$ to check your final solution: $y(t=1)=0.2295$.




%%%%%%%%%%%%%%%%%%%%%%%%%%%%%%%%
\newpage
%%%%%%%%%%%%%%%%%%%%%%%%%%%%%%%%
\section{An exponential function}

\subsection*{Challenge}
Solve

\begin{equation}
    y''+5y'+4y=100e^{-2t}
\end{equation}

for $y$, given initial conditions $y(0)=-1$ and $y'(0)=0$. Since the algebra gets very messy, you may use the following equation to help you:
\begin{equation}
    \frac{-s^2-7s+90}{(s+1)(s+2)(s+4)} = \frac{32}{s+1} - \frac{50}{s+2} + \frac{17}{s+4}
\end{equation}

\subsection*{Solution}
Substitute $t=1$ to check your final solution: $y(1)=5.32$.




%%%%%%%%%%%%%%%%%%%%%%%%%%%%%%%%
\newpage
%%%%%%%%%%%%%%%%%%%%%%%%%%%%%%%%
\section{A unit step}

\subsection*{Comment}
In past challenges we studied the Laplace transform for $U_c f(t-c)$. So if $f(t)=t$ we must evaluate for $f(t-c)=t-c$. In the challenge here, we effectively have $f(t)=1$ and since ``1'' doesn't depend on $t$, $t-c$ doesn't do anything to the ``function''.

This challenge is interesting because unlike previous challenges, it is the first challenge where we really have no other option but to use the Laplace transform method, and so you can appreciate its power. In this challenge, we have a 2nd-order homogeneous equation (unforced oscillation) until $t=5$ when we apply a constant force. You will find your answer leads to a constant oscillation. But how can it lead to a constant oscillation if we are constantly applying a force? Shouldn't the oscillation slowly increase in magnitude due to the energy that is being added to the system from the constant force being applied? The answer is of course no: we take just as much energy out of the system when the velocity is in the opposite direction to the force as we add to the system when the velocity is in the same direction as the applied force.

One important point to note is that the inverse Laplace transform of $e^{-cs} s/(s^2+a^2)$ is $\lap{U_c Cos(a[t-c])}$ (not $\lap{U_c Cos([at-c])}$).

\subsection*{Challenge}
Solve

\begin{equation}
    y''+2y=U_5
\end{equation}

for $y$, given initial conditions $y(0)=0$ and $y'(0)=0$. 

\subsection*{Solution}

$y(t=6)=0.42$

Note that for $t<5$, the solution is zero. This is because there was no initial velocity and no initial acceleration, so there was no motion until a forcing was applied in terms of a constant force of ``1'' from $t=5$. If either of these had been non-zero, we would have had a non-zero value for $t<5$!

Optionally, you can try setting the initial conditions to non-zero values to see the effect this has on the final solution.




%%%%%%%%%%%%%%%%%%%%%%%%%%%%%%%%
\newpage
%%%%%%%%%%%%%%%%%%%%%%%%%%%%%%%%
\section{A sudden impulse}

\subsection*{Comment}
Here the system is stationary until $t=5$ when, instead of applying a constant force, we ``kick'' the system to start the oscillation. Thus you should expect your answer to reflect physics such as this.

\subsection*{Challenge}
Solve

\begin{equation}
    y''+2y=\delta(t-5)
\end{equation}

for $y$, given initial conditions $y(0)=0$ and $y'(0)=0$. 

\subsection*{Solution}

$y(6)=0.698$

Note how we have a simple oscillation after $t=5$, and nothing before it.

\chapter{Systems of ODE's}
\section{Homogeneous vs non-homogeneous}

\subsection*{Resources}
\begin{itemize}
    \item Page 1 of the PDF \url{http://www.math.psu.edu/tseng/class/Math251/Notes-LinearSystems.pdf}
\end{itemize}

\subsection*{Challenge}
Separately add the points of the following \emph{homogeneous} and \emph{non-homogeneous} ODE systems:

1 point:
$\displaystyle
\left(
    \begin{array}{c}
        x_1' \\
        x_2' \\
        x_3'
    \end{array}
\right)
=
\left(
    \begin{array}{ccc}
        1 & 2 & 3 \\
        4 & 5 & 6 \\
        7 & 8 & 9
    \end{array}
\right)
\left(
    \begin{array}{c}
        x_1 \\
        x_2 \\
        x_3
    \end{array}
\right)
+
\left(
    \begin{array}{c}
        0 \\
        0 \\
        0
    \end{array}
\right)
$

2 points:
$\displaystyle
\left(
    \begin{array}{c}
        x_1' \\
        x_2' \\
        x_3'
    \end{array}
\right)
=
\left(
    \begin{array}{ccc}
        1 & 2 & 3 \\
        4 & 5 & 6 \\
        7 & 8 & 9
    \end{array}
\right)
\left(
    \begin{array}{c}
        x_1 \\
        x_2 \\
        x_3
    \end{array}
\right)
+
\left(
    \begin{array}{c}
        Cos(t) \\
        0 \\
        0
    \end{array}
\right)$

4 points:
$\displaystyle
\left(
    \begin{array}{c}
        x_1' \\
        x_2' \\
        x_3'
    \end{array}
\right)
=
\left(
    \begin{array}{ccc}
        1 & 2 & 3 \\
        4 & 5 & 6 \\
        7 & 8 & 9
    \end{array}
\right)
\left(
    \begin{array}{c}
        x_1 \\
        x_2 \\
        x_3
    \end{array}
\right)
+
\left(
    \begin{array}{c}
        Cos(t) \\
        Sin(t) \\
        0
    \end{array}
\right)$

8 points:
$\displaystyle
\left(
    \begin{array}{c}
        x_1' \\
        x_2' \\
        x_3'
    \end{array}
\right)
=
\left(
    \begin{array}{ccc}
        1 & 2 & 3 \\
        4 & 5 & 6 \\
        7 & 8 & 9
    \end{array}
\right)
\left(
    \begin{array}{c}
        x_1 \\
        x_2 \\
        x_3
    \end{array}
\right)
+
\left(
    \begin{array}{c}
        Cos(t) \\
        Sin(t) \\
        Tan(t)
    \end{array}
\right)$

\subsection*{Solution}
Homogeneous: \hash{hhh}{106c67} \\
Non-homogeneous: \hash{iii}{d51f57}




%%%%%%%%%%%%%%%%%%%%%%%%%%%%%%%%
\newpage
%%%%%%%%%%%%%%%%%%%%%%%%%%%%%%%%
\section{Basis for creating a system of equations from a single ODE}
\label{sec:systembasis}

\subsection*{Resources}
\begin{itemize}
    \item Pages 1-4 of the PDF \url{http://www.math.psu.edu/tseng/class/Math251/Notes-LinearSystems.pdf} 
\end{itemize}

\subsection*{Comment}
\emph{Note that the notation $y^{(2)}$ means ``the 2nd differential of y'' while the notation $y^2$ (without the brackets around the $2$) means ``y-squared''.}

Considering the general form of an nth-order linear equation,
\begin{equation}
    a_n y^{(n)} + a_{n-1} y^{(n-1)} + \cdots + a_1 y^{(1)} + a_0 y = g(t)
\end{equation}
we substitute $x_1=y$, $x_2=y'$, \ldots, $x_n=y^{(n-1)}$ and $x_n'=y^{(n)}$.

When replacing a $y$-term by an $x$ term, the $n$ in $x_n$ corresponds to one more than the number of times $y$ is differentiated. So $x_{n+1}$ corresponds to $y$ being differentiated $n$ times and $x_n$ corresponds to $y$ being differentiated $n-1$ times. So $x_2$ corresponds to $y^{(1)}$ (differentiated 1 time) and $x_1$ corresponds to $y$ (differentiated 0 times).

Note that $x_n'$ is one more differential than $x_n$, so $x_n'$ corresponds to $(y^{(n-1)})' = y^{(n)}$.
So the $n$ in $x_n'$ corresponds to the number of times $y$ is differentiated (ie, $y^{(n)}$).

The examples on page 3 are clearer after reading page 4, so I encourage you to read page 4 before considering the examples.

Considering example (II) on page 3, you are given the equation
\begin{equation}
    y''' - 2y'' + 3y' - 4y = 0
\end{equation}

To add a more detailed explanation to that found in the PDF: First re-write the ODE in terms of $x$ and $x'$. Note that there is no ``$x_0'$'' so we just write it as $x_1$ in both equations.
\begin{align}
    x_4 - 2 x_3 + 3 x_2 - 4 x_1 &= 0 \label{eq:xs} \\
    x_3' - 2 x_2' + 3 x_1' - 4 x_1 &= 0 \label{eq:xprimes}
\end{align}

Our aim is to write the system of equations in the form $\bm{x'} = \bm{A}\bm{x}$. Note that there is no ``$x_4'$'' in our equations, so the largest value of $n$ in $x_n'$ will be 3 (ie, $x_3'$).
\begin{equation}
    \left(
        \begin{array}{c}
            x_1' \\
            x_2' \\
            x_3'
        \end{array}
    \right)
    =
    \left(
        \begin{array}{ccc}
            ? & ? & ? \\
            ? & ? & ? \\
            ? & ? & ?
        \end{array}
    \right)
    \left(
        \begin{array}{c}
            x_1 \\
            x_2 \\
            x_3
        \end{array}
    \right)
\end{equation}
where the question marks are values that we have to find.

By direct comparison of equations \ref{eq:xs} and \ref{eq:xprimes} we know that $x_1' = x_2$ which can be written as $x_1' = 0 x_1 + 1 x_2 + 0 x_3$ yielding the first line in the matrix $\bm{A}$:
\begin{equation}
    \left(
        \begin{array}{c}
            x_1' \\
            x_2' \\
            x_3'
        \end{array}
    \right)
    =
    \left(
        \begin{array}{ccc}
            0 & 1 & 0 \\
            ? & ? & ? \\
            ? & ? & ?
        \end{array}
    \right)
    \left(
        \begin{array}{c}
            x_1 \\
            x_2 \\
            x_3
        \end{array}
    \right)
\end{equation}

We can then proceed to do $x_2$ in a similar fashion:
\begin{equation}
    \left(
        \begin{array}{c}
            x_1' \\
            x_2' \\
            x_3'
        \end{array}
    \right)
    =
    \left(
        \begin{array}{ccc}
            0 & 1 & 0 \\
            0 & 0 & 1 \\
            ? & ? & ?
        \end{array}
    \right)
    \left(
        \begin{array}{c}
            x_1 \\
            x_2 \\
            x_3
        \end{array}
    \right)
\end{equation}

In order to express $x_3'$ in the above matrix form, we need it in terms of $x_1$, $x_2$ and $x_3$ rather than $x_4$, so instead of direct comparison, we swap $x_4$ for $x_3'$ in equation \ref{eq:xs} to read
\begin{equation}
    x_3' - 2 x_3 + 3 x_2 - 4 x_1 = 0
\end{equation}
and then isolate $x_3'$ to read $x_3' = 4 x_1 - 3 x_2 + 2 x_3$ yielding the final form of our systems of equations
\begin{equation}
    \left(
        \begin{array}{c}
            x_1' \\
            x_2' \\
            x_3'
        \end{array}
    \right)
    =
    \left(
        \begin{array}{ccc}
            0 & 1 & 0 \\
            0 & 0 & 1 \\
            4 & -3 & 2
        \end{array}
    \right)
    \left(
        \begin{array}{c}
            x_1 \\
            x_2 \\
            x_3
        \end{array}
    \right)
\end{equation}

Note that this is only considering a homogeneous equation. If it is non-homogeneous, you will have an extra term in the final step and will need a matrix of the form  $\bm{x'} = \bm{A}\bm{x} + \bm{g}$ as shown in the answer to exercise 4(b) on page 5 of the PDF.

So why do we want to do this? Well, notice that in this example we started with a complicated 3rd-order ODE and reduced it into 3 1st-order ODE's. Similarly, if we started with a 2nd-order ODE, we could reduce the equation to 2 1st-order ODE's. In general, for an nth-order ODE we can reduce it to $n$ 1st-order ODE's. If we can then learn how to solve simultanious sets of 1st-order ODE's, we have a powerful method of increasing our understanding (and even solving) difficult higher-order ODE's. 

Similarly, if you are given a system of 2 1st-order ODE's, you can know that it can form a single 2nd-order ODE. 

\subsection*{Challenge}
Write the following ODE's in matrix form:

1) $2 y'' + 4 y' - 6 y = 0$

2) $y'' + y = Cos(t)$

Complete exercises 1 and 2 on page 5 of the PDF.

\subsection*{Solutions}
To check your answers, sum the values of all the terms in your matrix $\bm{A}$.

1) 2

2) 0 (remember there is also a $+\bm{g}$ column-vector added to $\bm{A}\bm{x}$ too)

The answers to the PDF exercises are shown on page 5 of the PDF. Perhaps obviously, since you will not have the answers in a real-life/exam environment, please don't review each answer until completion. If you get stuck, be sure to review your notes (especially the worked-examples in the PDF) rather than the answers, to facilitate deep learning. 




%%%%%%%%%%%%%%%%%%%%%%%%%%%%%%%%
\newpage
%%%%%%%%%%%%%%%%%%%%%%%%%%%%%%%%
%\input{matricies}
\section{Matricies}

\subsection*{Resources}
\begin{itemize}
    \item PDF: Pages 6-17 of the PDF \url{http://www.math.psu.edu/tseng/class/Math251/Notes-LinearSystems.pdf}
\end{itemize}

\subsection*{Comment}
It is worth spending some time getting comfortable with manipulating matricies, since this is an indispensible basis for the work that is about to follow. The PDF gives a quick introduction to matricies. For a more thorough introduction, the Khan Academy playlist on linear algebra [1] is excellent, although beyond the scope of this course.

One note to deal with any confusion arising with regard to eigenvectors with matricies with zeros. For $(A-rI)$ equal to something like
\begin{equation}
\left(
    \begin{array}{cc}
        0 & 0 \\
        1 & 2
    \end{array}
\right)
\end{equation}
the top row can be ignored since any $x_1$ and $x_2$ will satisfy the top row.

Similarly, for a case such as
\begin{equation}
\left(
    \begin{array}{cc}
        2 & 0 \\
        2 & 0
    \end{array}
\right)
\end{equation}
you will have
\begin{align}
    2 x_1 + 0 x_2 &= 0 \\
    2 x_1 &= 0 \\
    x_1 &= 0
\end{align}
which is satisfied by
\begin{equation}
\left(
    \begin{array}{c}
        0 \\
        1
    \end{array}
\right)
\end{equation}
(where the $1$ could in principle be any number, but is the minimum integer that satisfies the condition.)

Finally, note that $(A-rI) = ((a, b), (c, d))$ will give you two equivalent formulas $a x_1 + b x_2 = 0$ and $c x_1 + b x_2 = 0$, even if they may appear different on first glance. If you want, you can prove to yourself that they are the same by multiplying the bottom row by $a/c$.

\vspace{0.2cm}
\noindent [1] \url{https://www.khanacademy.org/math/linear-algebra/alternate-bases}

\subsection*{Challenges}
Complete exersizes 1, 2, 3, 4 (I and II only) and 5 on page 18 of the PDF.

\subsection*{Solutions}
The answers to the PDF exercises are shown on page 18 of the PDF. Perhaps obviously, since you will not have the answers in a real-life/exam environment, please don't review each answer until completion. If you get stuck, be sure to review your notes (especially the worked-examples in the PDF) rather than the answers, to facilitate deep learning. 

The solution to question 1 above can be found on the next page.




%%%%%%%%%%%%%%%%%%%%%%%%%%%%%%%%
\newpage
%%%%%%%%%%%%%%%%%%%%%%%%%%%%%%%%
\section{Eigenvector equivalence}

\subsection*{Comment}
Considering the matrix
\begin{equation}
    A = \left(
        \begin{array}{cc}
            1 & 2 \\
            -3 & -4
        \end{array}
    \right)
\end{equation}
The eigvenvalues are -2 and -1. Considering the eigenvalue -2, 
\begin{equation}
    A - Ir = \left(
        \begin{array}{cc}
            3 & 2 \\
            -3 & -2
        \end{array}
    \right)
\end{equation}
To determine the eigenvector we can either take the top or bottom row in the calculation $(A - Ir)x = 0$.
The top and bottom row appear with different numbers but it is easy to see that they yield multiples of the same eigenvector and are therefore equivalent.

Complex eigenvectors are no different, but it can sometimes be hard to see that they are indeed equivalent.

\subsection*{Challenge}
Show that the equation $(A - Ir)\bm{x} = {0}$, where
\begin{equation}
     A-Ir = \left(
        \begin{array}{cc}
            -3 -3i & 6 \\
            -3 & 3-3i
        \end{array}
    \right)
\end{equation}
yields the same eigenvector, irrespective of whether you calculate the eigenvector using the top or bottom row of $(A-Ir)$. You may find that one of the representations of the eigenvectors looks like $(i-1,1)$.

\subsection*{Solutions}
You should be able to generate two eigenvectors by using the top and bottom rows of the $A-Ir$ matrix, and show that they are infact the same eigenvector by multiplying by an equivalent (imaginary) number. Please discuss with your partner or the teacher in class if you have trouble.



%%%%%%%%%%%%%%%%%%%%%%%%%%%%%%%%
\newpage
%%%%%%%%%%%%%%%%%%%%%%%%%%%%%%%%
\section{Solving systems of ODE's}
\label{sec:systemsolving}

\subsection*{Resources}
\begin{itemize}
    \item Pages 6-31 of the PDF \url{http://www.math.psu.edu/tseng/class/Math251/Notes-LinearSystems.pdf} 
\end{itemize}

\subsection*{Challenge}
Complete at least exercises 1-10 on page 32-33 of the PDF.

\subsection*{Solutions}
It might not be clear to you why solutions involve vectors and what this means physically, but for now, please just get used to solving equations in this fashion.

The answers are shown on page 33-34 of the PDF. Perhaps obviously, since you will not have the answers in a real-life/exam environment, please don't review each answer until completion. If you get stuck, be sure to review your notes (especially the worked-examples in the PDF) rather than the answers, to facilitate deep learning. 

%NT: Add something about intersection of graphs? https://en.wikipedia.org/wiki/System_of_linear_equations
% Good resources: http://www.math.psu.edu/tseng/class/Math251/




%%%%%%%%%%%%%%%%%%%%%%%%%%%%%%%%
\newpage
%%%%%%%%%%%%%%%%%%%%%%%%%%%%%%%%
\section{Graphs of system solutions}

\subsection*{Resources}
In the previous challenge you determined $x_1$ and $x_2$ with solutions such as
\begin{equation}
    \bm{x} = \matrixcrr{x_1}{x_2} = c_1 \matrixcrr{-1}{6} e^{-6t} + c_2 \matrixcrr{1}{1} e^t
\end{equation}

or written another way:
\begin{align}
    x_1 &= -c_1 e^{-6t} + c_2 e^t \label{eqn:sysgraphx1} \\
    x_2 &= 6 c_1 e^{-6t} + c_2 e^t \label{eqn:sysgraphx2} 
\end{align}

This particular system arose from a 2nd-order differential equation:
\begin{equation}
    y'' + 5y' - 6y = 0 \label{eqn:sys2ndoode}
\end{equation}

we have learned in challenge \ref{sec:systembasis} that this 2nd-order equation can be written in terms of $x$:

\begin{equation}
    x_3 + 5x_2 - 6 x_1 = 0
\end{equation}

Thus we remember that $x_1 = y$ and $x_2 = y'$, allowing equations \ref{eqn:sysgraphx1} and \ref{eqn:sysgraphx2} to be written as
\begin{align}
    y &= -c_1 e^{-6t} + c_2 e^t \label{eqn:syspos} \\
    y' &= 6 c_1 e^{-6t} + c_2 e^t \label{eqn:sysvel} 
\end{align}

Perhaps, for example, the original 2nd-order ODE (equation \ref{eqn:sys2ndoode}) represented the position of an atom on an axis with respect to time. Then equation \ref{eqn:syspos} represents position at time $t$ while equation \ref{eqn:sysvel} represents the velocity (or more commonly, when multiplied by the mass, represents the momentum).

Thus the graph represents the variation of momentum (velocity) with position, called the ``phase-space'' of the system. A specific trajectory can be followed given boundary conditions that determine the starting condition. For example, if the particle at time $t=0$ is known to have position $y=1$ and velocity $y'=2$ we can impose the boundary condition
\begin{equation}
    \bm{x}(0) = \matrixcrr{1}{2}
\end{equation}
to determine the coefficients $c_1$ and $c_2$ and obtain a unique trajectory.

We can then plot the phase-space for various boundary conditions. In the graph below, we show examples where $c_1 = c_2 = \{0, 0.5, 1, 1.5, 2\}$:

\includegraphics[scale=0.7]{phase_space_1.png}

You can note that as $t$ increases, the term $e^{-6t}$ goes to zero leaving the $e^t$ dominant, and since this features in both $y$ and $y'$, you get $y \propto y'$ for large $t$. % NT: Check this: As expected for this negatively-damped equation, both the velocity and position increase rapidly with time.

The examples we are considering here are relatively simple, however this can be used to identify complex and chaotic phenomena visually. For example, considering a pendulum gently swinging backwards and forwards, it is possible to trace out the phase-space as shown here:

\includegraphics[scale=0.4]{pendulum_gentle.png}\\
\emph{\small{Source: \url{https://commons.wikimedia.org/wiki/File:Pendulum_phase_portrait_illustration.svg}, Wikipedia user Krishnavedala}}

If you increase the speed of the pendulum, at some critical point, instead of swinging back to the original position it will start whirring round and round. Expressed in terms of vertical angle and angular velocity, the graph becomes:

\includegraphics[scale=1]{pendulum_fullphase.png}\\
\emph{\small{Source: \url{https://commons.wikimedia.org/wiki/File:Pendulumphase.png}}}

At low velocities the pendulum swings back and forth (blue circles, angular velocity both positive and negative), but at high velocities, the angular velocity stays positive (or negative) and the pendulum whirs round and round in one direction (blue wavy lines). Note that position $\theta = \pi$ is when the rigid pendulum is pointing exactly upwards. So with no momentum it is stationary here, albeit unstable, because with a tiny velocity it will perform a full loop, slowing (but not stopping) as it reaches the top again.

\subsection*{Challenge}
1. The graphs below represent the solutions to the exercises 1-5 in challenge \ref{sec:systemsolving}.
Place the graphs below in the same order as exercises 1-5. Note that in order to maintain clarity, the graphs are not necessarily plotted over the same time interval $t$.

\begin{tabular}{cc}
    \includegraphics[scale=0.6]{phase_space_a.png} &
    \includegraphics[scale=0.6]{phase_space_b.png} \\
    \includegraphics[scale=0.6]{phase_space_c.png} &
    \includegraphics[scale=0.5]{phase_space_d.png} \\
    \includegraphics[scale=0.6]{phase_space_e.png} &
\end{tabular}
% NT: Add graphs that have the same starting position so students need to check the shape of the graph, as well as the t=0 position

\vspace{1cm}
2. Considering the graph shown earlier of angular momentum vs angle for a rigid pendulum, add the points of the following true statements:

\textbf{1 point} An initial angular velocity of 1 unit results in whirring circular motion irrespective of the starting angle.

\textbf{2 points} An initial angular velocity of -2.5 units results in whirring circular motion irrespective of the starting angle.

\textbf{4 points} An initial angle of $\pi/2$ combined with an angular velocity of 1 unit results in periodic swinging motion.

\textbf{8 points} An initial angle of $\pi/2$ combined with an angular velocity of 1 unit results in circular whirring motion.

\textbf{16 points} An initial angle of $0$ combined with an angular velocity of 0 units results in periodic swinging motion.

\textbf{32 points} An initial angle of $0$ combined with an angular velocity of 0 units results in a stationary system.

\textbf{64 points} An initial angle of $\pi$ combined with an angular velocity of 0 units results in a stationary system.

\textbf{128 points} An initial angle of $\pi/2$ combined with an angular velocity of 0 units results in a stationary system.

\textbf{256 points} An initial angular velocity of 3 units results in whirring circular motion in the same direction as an initial angular velocity of -3 units.

\textbf{512 points} An initial angular velocity of 3 units results in whirring circular motion in the opposite direction as an initial angular velocity of -3 units.

\subsection*{Solutions}

1. (eg, ``abcde'') \hash{jjj}{778bbb}


2. (enter number to 2 decimal places, as usual) \hash{kkk}{7febe0}


% NT: bifurcation? https://en.wikipedia.org/wiki/Hopf_bifurcation
% Other Pendulum resources of interest: http://math.stackexchange.com/questions/666806/nonlinear-pendulum and https://www.youtube.com/watch?v=6KSXLGHsrrs

\chapter{Numerical methods}
\section{System set-up}

\subsection*{Comment}
Numerical methods of solving ODE's are common. Here we will consider Euler and Runge-Kutta approaches.

\subsection*{Challenge}

Considering the following ODE:

\begin{equation}
    \dot{v} = g - v^2
\end{equation}

1. Sketch a direction field to show the behaviour of solutions to this ODE. In particular, what values of $v$ will lead to stable solutions?

2. What type of ODE is this?

\subsection*{Solutions}
1. (Stable)\\
\soltwodp{a}{4ac0c3}

1. (Unstable)\\
\soltwodp{b}{9ed192}

2. Please compare your solution with your partner or ask the teacher.




%%%%%%%%%%%%%%%%%%%%%%%%%%%%%%%%
\newpage
%%%%%%%%%%%%%%%%%%%%%%%%%%%%%%%%
\section{Tangent lines}

\subsection*{Resource}
\begin{itemize}
    \item Chapter 3: \url{https://raw.githubusercontent.com/kriskissel/ConceptsODE/master/main.pdf}
\end{itemize}

\subsection*{Challenge}
1. Given that $y(t)$ satisfies the equation $y' = y^3 + 3t$ subject to $y(1) = 2$, find $y'(1)$ without solving the differential equation and obtain the equation of the tangent to the curve $y(t)$ at the point (1,2).

2. Use the tangent line to estimte the value at $t = 1.5$.

\subsection*{Solution}
2. $7.5$




%%%%%%%%%%%%%%%%%%%%%%%%%%%%%%%%
\newpage
%%%%%%%%%%%%%%%%%%%%%%%%%%%%%%%%
\section{Euler's method}
\label{sec:euler}

\subsection*{Resource}
\begin{itemize}
    \item Chapter 3: \url{https://raw.githubusercontent.com/kriskissel/ConceptsODE/master/main.pdf}
\end{itemize}

\subsection*{Challenge}
Given that $v(t)$ satisfies the relation $v' = g - v^2$, assuming an initial value of $v(0)=0$, using Euler's method estimate $v(1)$ using step sizes of

1. $\Delta t = 1/2$\\
2. $\Delta t = 1/4$

Explain the difference in the behaviour with the different step sizes. It may be helpful to draw a graph.

\subsection*{Solution}
1. $v(1) = -2.22$\\
2. $v(1) = 3.22$



%%%%%%%%%%%%%%%%%%%%%%%%%%%%%%%%
\newpage
%%%%%%%%%%%%%%%%%%%%%%%%%%%%%%%%
\section{4th-order Runge-Kutta}

\subsection*{Resource}
\begin{itemize}
    \item Chapter 3: \url{https://raw.githubusercontent.com/kriskissel/ConceptsODE/master/main.pdf}
\end{itemize}

\subsection*{Comment}
Derivation of the Runge-Kutta method is beyond this course, however there are many resources online going into more detail. This video-series is nice:
\begin{enumerate}
    \item \url{https://www.youtube.com/watch?v=b-OSyxOpxKc}
    \item \url{https://www.youtube.com/watch?v=JySrVHRmqfU}
    \item \url{https://www.youtube.com/watch?v=iS3hsHGY1Ok}
    \item \url{https://www.youtube.com/watch?v=wr3-dWoxiY4}
\end{enumerate}

\subsection*{Challenge}
1. Using the same function as challenge \ref{sec:euler}, estimate $v(1/2)$ using the Runge-Kutta method and a step-size of $\Delta t = \frac{1}{4}$.

2. Compare your answer to $v(1/2)$ obtained using the Euler method with the same step-size. How does the behaviour differ?

\subsection*{Solution}
1. $v(1/2) = 2.99$

2. Please compare your answer with your partner or check with the teacher.




%%%%%%%%%%%%%%%%%%%%%%%%%%%%%%%%
\newpage
%%%%%%%%%%%%%%%%%%%%%%%%%%%%%%%%
\section{Runge-Kutta vs Euler method}

\subsection*{Resource}
\begin{itemize}
    \item Chapter 3: \url{https://raw.githubusercontent.com/kriskissel/ConceptsODE/master/main.pdf}
\end{itemize}

\subsection*{Challenge}
Briefly explain the advantages that the Runge-Kutta method has over the Euler method.

\subsection*{Solution}
Please compare your answer with your partner or check with the teacher.

\appendix
%\chapter{Extra challenges}
%\include{extra_challenges}
\chapter{Solutions}
\section{Challenge \ref{sec:systemsolvingchallenges}}
\label{sec:systemsolvingsols}

2. $\displaystyle x = C_1 \matrixcrr{7}{-5} e^{-3t} + C_2 \matrixcrr{1}{-1} e^{-5t}$

3. $\displaystyle x = C_1 \matrixcrr{\cos 3t + \sin 3t}{\cos 3t} + C_2 \matrixcrr{- \cos 3t + \sin 3t}{\sin 3t}$

4. $\displaystyle x = C_1 \matrixcrr{2}{1} e^{6t} + C_2 \left( \matrixcrr{2}{1} t e^{6t} + \matrixcrr{1}{0} e^{6t} \right)$

%\chapter{Mid-term exam questions}
%\section{}

Solve the following ODE for $y$ given the condition $y(3)=9e^9$.

\begin{equation}
    \frac{x}{y} \frac{dy}{dx} - 1 = x^3
\end{equation}





\section{}

The following equation is an autonamous equation:

\begin{equation}
    y'=\frac{y^2}{5}(1-\frac{y}{5})
\end{equation}
1. What key property does an autonamous equation have?


2. Determine the points of equilibrium and their stabilities.





\section{}

Solve the following 2nd-order ODE's for $y$, and state what sort of damping they correspond to:

\begin{equation}
    y'' + 5 y' + 4y = 0 % Real
\end{equation}


\begin{equation}
    y'' + 4 y' + 4 y = 0 % Equal
\end{equation}


\begin{equation}
    y'' + 3 y' + 4 y = 0 % Complex
\end{equation}




\section{}

Solve the following differential equation for $y$:

\begin{equation}
    3 x^2 y + 2 x y + y^3 + (x^2 + y^2) y' = 0
\end{equation}




\vspace{2cm}

\emph{Solutions can be found on the following page.}
\newpage

\section{Solutions}

\textbf{Question 1}

$y=3xe^{x^3/3}$

\textbf{Question 2}

1. $y'=f(y)$

2. $y=0$ (semi-stable), $y=5$ (stable)

\textbf{Question 3}

$y(t)=C_1 e^{-t} + C_2 e^{-4t}$, Overdamped

$y(t)=C_1 e^{-2t} + C_2 t e^{-2t}$, Critically-damped

$y(t)=C_1 e^{-3t/2} Cos(\sqrt{7}t/2) + C_2 e^{-3t/2} Sin(\sqrt{7}t/2)$, Under-damped

\textbf{Question 4}

$C = yx^2e^{3x} + \frac{1}{3} y^3 e^{3x}$


% Series solutions https://www.youtube.com/playlist?list=PLj7p5OoL6vGxVBxyLWLQHOCfFTIY_if5C
% Euler http://nbviewer.jupyter.org/github/engineersCode/EngComp/blob/master/modules/3_flyatchange/3_Get_Oscillations.ipynb
%       https://hub.mybinder.org/user/engineerscode-engcomp-7t4hdbpy/tree/modules/3_flyatchange 

\end{document}

% Interesting resources about the need for exponential and sinusoidal solutions
% https://www.youtube.com/watch?v=ZGPtPkTft8g (includes Laplace)
% https://math.stackexchange.com/questions/573581/why-is-the-formal-solution-to-a-linear-differential-equation-of-exponential-form
% https://math.stackexchange.com/questions/2502533/when-solving-a-differential-equation-why-do-we-always-start-with-guessing-the-s

