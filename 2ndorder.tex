\section{Hooke's law}

\subsection*{Resources}

\includegraphics[scale=0.5]{hook.png}\\
\emph{(\href{http://hyperphysics.phy-astr.gsu.edu/hbase/imgmec/hook.gif}{Image} from HyperPhysics by Rod Nave, Georgia State University)}

\subsection*{Challenge}
2nd-order differential equations deal with oscillations.

Considering Hooke's law, what are $A$ and $C$ in the following equation?
\begin{equation}
    A x'' + C x = 0
\end{equation}

To check your answer, substitute a mass of \SI{2}{kg} and spring-constant of \SI{3}{kg/s^2} as appropriate.

\subsection*{Solution}
Enter only numerical values without units such as kg.

A: \hash{gg}{4e5fe6}

C: \hash{hh}{6a7015}




%%%%%%%%%%%%%%%%%%%%%%%%%%%%%%%%%
\newpage
%%%%%%%%%%%%%%%%%%%%%%%%%%%%%%%%%
\section{Exponentials and trigonometry}

\subsection*{Resources}
\begin{itemize}
    \item Text: \url{https://www.phy.duke.edu/~rgb/Class/phy51/phy51/node15.html}
\end{itemize}

\subsection*{Challenge}
Write $sin(x)$ and $cos(x)$ in exponential form.

\subsection*{Solution}

Check your answer with someone if you are unsure.


\timebox




%%%%%%%%%%%%%%%%%%%%%%%%%%%%%%%%%
\newpage
%%%%%%%%%%%%%%%%%%%%%%%%%%%%%%%%%
\section{Characteristic equation: understanding}

\subsection*{Resources}
\begin{itemize}
    \item Book (\url{http://tutorial.math.lamar.edu/getfile.aspx?file=B,1,N}) from page 111.
\end{itemize}

\subsection*{Comment}
A homogeneous (ie, equal to zero) second-order differential equation typically takes the form:

\begin{equation}
    A \frac{d^2y}{dt^2} + B \frac{dy}{dt} + C y = 0
\end{equation}

The first (A) term describes acceleration, while the third (C) term is the force-constant term (something like the ``stiffness'' of the spring). The second (B) term could describe a frictional force that is proportional to the velocity ($dy/dt$). Due to its relation with oscillation (and by extension, sines and cosines which can be expressed in terms of exponentials) we can typically assume an exponential-form solution to the differential equation.

\subsection*{Challenge}
Show that, assuming that all solutions to a 2nd-order differential equation of the form above will have solutions $y(t)=e^{rt}$, the value of $r$ can in principle be determined by solving the following a quadratic equation of the form
\begin{equation}
    A r^2 + Br + C = 0
\end{equation}

\subsection*{Solution}
If you are unsure of your derivation, please ask someone.

\timebox




%%%%%%%%%%%%%%%%%%%%%%%%%%%%%%%%%
\newpage
%%%%%%%%%%%%%%%%%%%%%%%%%%%%%%%%%
\section{Characteristic equation: roots}

\subsection*{Resources}
\begin{itemize}
    \item Book (\url{http://tutorial.math.lamar.edu/getfile.aspx?file=B,1,N}) from page 111.
\end{itemize}

\subsection*{Challenge}
Sum the points of the differential equations that have characteristic equations with
\begin{itemize}
    \item Real, distinct roots
    \item Complex roots
    \item Equal roots
\end{itemize}

1 point: $\displaystyle -3 y'' - 5 y' + 2 y = 0$ % C

2 points: $\displaystyle 3 y'' - 4 y' + 3 y = 0$ % E

4 points: $\displaystyle 3 y'' - 6 y' + 3 y = 0$ % B

8 points: $\displaystyle 3 y'' - 5 y' + 2 y = 0$ % F

16 points: $\displaystyle 3 y'' - 5 y' + 4 y = 0$ % D

32 points: $\displaystyle 3 y'' + 5 y' + 2 y = 0$ % A

\subsection*{Solution}

\begin{itemize}
    \item Real, distinct roots: \hash{ii}{064a6e}
    \item Complex roots: \hash{jj}{5cdb6c}
    \item Equal roots: \hash{kk}{70cd8f}
\end{itemize}

\timebox




%%%%%%%%%%%%%%%%%%%%%%%%%%%%%%%%%
\newpage
%%%%%%%%%%%%%%%%%%%%%%%%%%%%%%%%%
\section{Characteristic equation: real roots with positive B}

\subsection*{Resources}
\begin{itemize}
    \item Book (\url{http://tutorial.math.lamar.edu/getfile.aspx?file=B,1,N}) from page 116.
\end{itemize}

\subsection*{Challenge}
Solve the following 2nd-order differential equation that has real roots:

\begin{equation}
    \label{eq:ccrrpb}
    y'' + 3 y' + 2 y = 0
\end{equation}

with initial conditions $y(0)=5$ and $y'(0)=-8$.

To check your answer, substitute $t=1$ into the final expression.


\subsection*{Solution}
1.14 %\hash{mm}{9b9be5}




%%%%%%%%%%%%%%%%%%%%%%%%%%%%%%%%%
\newpage
%%%%%%%%%%%%%%%%%%%%%%%%%%%%%%%%%
\section{Characteristic equation: real roots with negative B}

\subsection*{Resources}
\begin{itemize}
    \item Book (\url{http://tutorial.math.lamar.edu/getfile.aspx?file=B,1,N}) from page 116.
\end{itemize}

\subsection*{Challenge}
Solve the following 2nd-order differential equation that has real roots. 

\begin{equation}
    y'' - 3 y' + 2 y = 0
\end{equation}

with initial conditions $y(0)=5$ and $y'(0)=-8$. Substitute $t=1$ into the final expression to check your answer.

Note that this equation is the same as equation \ref{eq:ccrrpb}, but simply the dampening (friction) term B has been changed from positive to negative.


\subsection*{Solution}
-47.13 %\hash{nn}{473835}

\timebox




%%%%%%%%%%%%%%%%%%%%%%%%%%%%%%%%%
\newpage
%%%%%%%%%%%%%%%%%%%%%%%%%%%%%%%%%
\section{Characteristic equation: B in equations with real roots}

\subsection*{Challenge}

\emph{(Note that there are two parts to this challenge.)}

1. Considering real root, sum the points of the following true statements:

Considering the equation

\begin{equation}
    A y'' + B y' + C y = 0
\end{equation}

1 point: Positive damping (positive B) leads to solutions with exponentials with positive exponents.

2 points: Positive damping (positive B) leads to solutions with exponentials with negative exponents.

4 points: Negative damping (negative B) leads to solutions with exponentials with positive exponents.

8 points: Negative damping (negative B) leads to solutions with exponentials with negative exponents.

16 points: Exponentials with positive exponents (eg, $e^{t}$) lead to exponential growth (instability).

32 points: Exponentials with negative exponents (eg, $e^{-t}$) lead to exponential growth (instability).

64 points: Exponentials with positive exponents (eg, $e^{t}$) lead to a damped signal (stability).

128 points: Exponentials with negative exponents (eg, $e^{-t}$) lead to damped signal (stability).

\vspace{2em}

2. Write a sentence summarising your understanding of the significance of having a positive or negative coefficient of $B$ when the roots are real.


\subsection*{Solution}
\hash{oo}{fa6adf}

\timebox




%%%%%%%%%%%%%%%%%%%%%%%%%%%%%%%%%
\newpage
%%%%%%%%%%%%%%%%%%%%%%%%%%%%%%%%%
\section{Characteristic equation: equal roots}

\subsection*{Resources}
\begin{itemize}
    \item Book (\url{http://tutorial.math.lamar.edu/getfile.aspx?file=B,1,N}) from page 125.
\end{itemize}

\subsection*{Comment}
It is not necessary to follow the full derivation in the suggested resource.

\subsection*{Challenge}
Solve the equation
\begin{equation}
    y'' - 2y' + y = 0
\end{equation}

To check your solution, substitute $t=1$ into the equation and assume $c_1 = c_2 = 1$.

\subsection*{Solution}
5.44 %\hash{pp}{ff7ca2}

\timebox




%%%%%%%%%%%%%%%%%%%%%%%%%%%%%%%%%
\newpage
%%%%%%%%%%%%%%%%%%%%%%%%%%%%%%%%%
\section{Characteristic equation: complex roots with B=0}

\subsection*{Resources}
\begin{itemize}
    \item Book (\url{http://tutorial.math.lamar.edu/getfile.aspx?file=B,1,N}) from page 120.
\end{itemize}

\subsection*{Challenge}
1. Assuming there is no damping term (ie, $B=0$) show that the roots for the differential equation
\begin{equation}
    A y'' + Cy = 0
\end{equation}
are $\pm i \sqrt{C/A}$.

2. Solve the following ODE:
\begin{equation}
    \label{eq:cecr}
    y'' + 4 \pi^2 y = 0
\end{equation}

To check your answer, assume integration constants of 1 and calculate $y(\pi/2)$.

\subsection*{Solution}
Solution to part 2: -1.33 %\hash{qq}{7eb2c9}

\timebox




%%%%%%%%%%%%%%%%%%%%%%%%%%%%%%%%%
\newpage
%%%%%%%%%%%%%%%%%%%%%%%%%%%%%%%%%
\section{Characteristic equation: complex roots with positive B}

\subsection*{Resources}
\begin{itemize}
    \item Book (\url{http://tutorial.math.lamar.edu/getfile.aspx?file=B,1,N}) from page 120.
\end{itemize}

\subsection*{Challenge}
Solve the following ODE:
\begin{equation}
    y'' + y' + y = 0
\end{equation}

To check your answer, assume integration constants of 1 and calculate $y(\pi/2)$.

\subsection*{Solution}
-0.61 %\hash{rr}{1d0cb5}

\timebox




%%%%%%%%%%%%%%%%%%%%%%%%%%%%%%%%%
\newpage
%%%%%%%%%%%%%%%%%%%%%%%%%%%%%%%%%
\section{Characteristic equation: complex roots with negative B}

\subsection*{Resources}
\begin{itemize}
    \item Book (\url{http://tutorial.math.lamar.edu/getfile.aspx?file=B,1,N}) from page 120.
\end{itemize}

\subsection*{Challenge}
Solve the following ODE:
\begin{equation}
    y'' - y' + y = 0
\end{equation}

To check your answer, assume integration constants of 1 and calculate $y(\pi/2)$.

\subsection*{Solution}
-2.92 %\hash{ss}{caf35b}

\timebox




%%%%%%%%%%%%%%%%%%%%%%%%%%%%%%%%%
\newpage
%%%%%%%%%%%%%%%%%%%%%%%%%%%%%%%%%
\section{Damping}
\label{sec:damping}

\subsection*{Resources}
\begin{itemize}
    \item Wikipedia: \url{https://en.wikipedia.org/wiki/Damping}
\end{itemize}

\subsection*{Challenge}
Of the 6 functions shown in the graph, place the 3 that correspond to over-damped, critically damped and under-damped in the order mentioned in this sentence.

\includegraphics[scale=0.75]{damping.png}

\subsection*{Solution}
(eg, ``abc'')

\hash{tt}{060b2a}

\timebox



%%%%%%%%%%%%%%%%%%%%%%%%%%%%%%%%%
\newpage
%%%%%%%%%%%%%%%%%%%%%%%%%%%%%%%%%
\section{Damping and 2nd-order differential equations}

\subsection*{Challenge}
1. The 6 functions shown in the graph in challenge \ref{sec:damping} may represent solutions of a 2nd-order differential equation. Place the solutions A-F in the order shown below. Note that one of the descriptions below is impossible, and you should ignore that one.

I. Solution of a 2nd-order differential equation with real roots and positive B.

II. Solution of a 2nd-order differential equation with real roots and negative B.

III. Solution of a 2nd-order differential equation with real roots and B=0.

IV. Solution of a 2nd-order differential equation with equal roots.

V. Solution of a 2nd-order differential equation with complex roots and B=0.

VI. Solution of a 2nd-order differential equation with complex roots and positive B.

VII. Solution of a 2nd-order differential equation with complex roots and negative B.

\vspace{1em}
2. Write one sentence stating why one of the above solutions is impossible.

\subsection*{Solution}
(eg, ``abcdef'')

\hash{uu}{a96870}

\timebox




%%%%%%%%%%%%%%%%%%%%%%%%%%%%%%%%%
\newpage
%%%%%%%%%%%%%%%%%%%%%%%%%%%%%%%%%
\section{The Wronskian}

\subsection*{Resources}
\begin{itemize}
    \item Book (\url{http://tutorial.math.lamar.edu/getfile.aspx?file=B,1,N}) from page 133.
\end{itemize}

\subsection*{Challenge}

\emph{Please write the following answers clearly and in a manner that can be easily shared with others in the class.}

1. What is meant by a ``fundamental set of solutions''?

2. Why is the final solution for real and complex roots always a sum of two terms?

3. Why do the two terms need to be linearly independent? What would happen if they were not linearly independent?

4. What is the ``Wronskian'', and what is the formula for its calculation?

5. How can the Wronskian be used to determine linear independence?

\subsection*{Solution}
Please read at least 1 other peer's solution and discuss any differences. The teacher will also help check your understanding.

\timebox




%%%%%%%%%%%%%%%%%%%%%%%%%%%%%%%%
\newpage
%%%%%%%%%%%%%%%%%%%%%%%%%%%%%%%%
\section{Characteristic equation: exercises}

\emph{(Note that if you encounter a square-root during your calculations such as $\sqrt{7}$, it is best to work with $\sqrt{7}$ rather than $2.65$ in order to maintain accuracy until the final step where you need to evaluate it. If the equation becomes too messy (eg $e^{(\sqrt{7}-1)/\sqrt{3}}$) you can always substitute $m=(\sqrt{7}-1)/\sqrt{3}$, etc, to make things clearer.)}

\subsection*{Challenge}
1. Determine $y(1)$ for the equation

\begin{equation}
    2 y''+8y'+y=0    
\end{equation}
given the initial conditions $y(0)=4$ and $y'(0)=3$.

2. Determine $y(0.2)$ for the equation

\begin{equation}
    2y''+4y'+2y=0
\end{equation}
given the initial conditions $y(0)=4$ and $y'(0)=2$.

3. Determine $y(0.1)$ for the equation

\begin{equation}
    4y''+3y'+y=0
\end{equation}
given the initial conditions $y(0)=6$ and $y'(0)=2$.


\subsection*{Solution}
1. 4.32 %\hash{vv}{f01192}

2. 4.26 %\hash{ww}{6f5d64}

3. 6.19 %\hash{xx}{c74f58}

\timebox




%%%%%%%%%%%%%%%%%%%%%%%%%%%%%%%%
\newpage
%%%%%%%%%%%%%%%%%%%%%%%%%%%%%%%%
\section{(C19,C20,C21,C22) Non-homogeneous equations: Method of undetermined coefficients}

\subsection*{Resources}
\begin{itemize}
    \item Video: All 4 Khan Academy videos starting at \url{https://www.khanacademy.org/math/differential-equations/second-order-differential-equations/undetermined-coefficients/v/undetermined-coefficients-1}
\end{itemize}

\subsection*{Comment}
The 2nd-order equations we were considering until now were homogeneous equations (ie, the RHS was zero). We can now build upon this to expand our ability to solve non-homogeneous equations (ie, where the RHS of the equation is non-zero).

The Khan Academy videos give an excellent initial introduction to the subject, and so please do take the time to view and take notes about all four videos in the series.

In the 4th video Mr Kahn describes about how it is possible to add solutions if there are multiple terms on the right. This occasionally causes confusion. Consider for example:

\begin{equation}
    y'' - 3y' - 4y = 2 \sin x
\end{equation}

This corresponds to the particular solution

\begin{equation}
    y_p = A \sin x + B \cos x
\end{equation}

A common point of confusion is about what to do in the case of something like
\begin{equation}
    \label{eq:particularconfusion}
    y'' - 3y' - 4y = 2 \sin x + 2 \cos x
\end{equation}

Should you just write $y_p = (A \sin x + B \cos x) + (C \sin x + D \cos x)$? After all, you have two terms in equation \ref{eq:particularconfusion} (ie, $2 \sin x$ and $2 \cos x$). You can note however that $A \sin x + C \sin x$ simplifies to $E \sin x$ where $E$ is just another constant (in this case $A+B$) so in the end you will be left with $y_p = E \sin x + F \cos x$. So while it may be clearer to explicitly calculate coefficients for every term on the RHS, in many cases the terms will simplify.


\subsection*{Challenge}
Find the general solution of the following non-homogeneous differential equations:

1. (C19) $y'' + 4y = 8$\\
2. (C20) $y'' + 4y = 8t^2 - 20t + 8$\\
3. (C21) $y'' + 4y = 5 \sin 3t - 5 \cos 3t$\\
4. (C22) $y'' + 4y = 24 e^{-2t}$

(please just rate the challenges on challenge-hub after you have determined the answer for each one)

\subsection*{Solution}
The solutions are contained in the list on the next page in no particular order. Your answers should match one of the solutions given. Please try to not look at the solutions before completing the questions, since this will facilitate deep understanding and reproduce a real-life/exam environment.
\newpage
$y = C_1 \cos 2t + C_2 \sin 2t + 3e^{-2t}$\\ %4
$y = C_1 \cos 2t + C_2 \sin 2t + 8e^{-2t}$\\
$y = C_1 \cos 2t + C_2 \sin 2t + 2t^2 - 5t + 1$\\ %2
$y = C_1 \cos 2t + C_2 \sin 2t + 3t^2 + t + 3$\\
$y = C_1 \cos 2t + C_2 \sin 2t + \cos 3t - \sin 3t$\\ %3
$y = C_1 \cos 2t + C_2 \sin 2t + 2$\\ %1
$y = C_1 \cos 2t + C_2 \sin 2t + 5$\\



%%%%%%%%%%%%%%%%%%%%%%%%%%%%%%%%
\newpage
%%%%%%%%%%%%%%%%%%%%%%%%%%%%%%%%
\section{(C23-28) Method of undetermined coefficients II}

\subsection*{Comment}
The following pages go into more detail than the videos, considering a greater range of cases. You may note that here the particular solution is denoted by $Y$ while Sal Khan denoted it as $y_p$ in the videos.

\emph{The following notes were developed by Zachary S. Tseng at Pennsylvania State University, USA (\url{http://www.math.psu.edu/tseng/}). Included here with kind permission.}

\includepdf[pages=-,pagecommand={},width=\textwidth,nup=1x1,frame=true]{External/undetermined.pdf}

\subsection*{Challenge}
The following challenges expand the range of problems to give you practise in a range of situations.

1. (C23) $y'' + 4y = 8 \cos 2t$\\
2. (C24) $y'' + 2y' = 2 te^{-t}$\\
3. (C25) $y'' + 2y' = 6 e^{-2t}$\\
4. (C26) $y'' + 2y' = 12 t^2$\\
5. (C27) $y'' - 6y' - 7y = 13 \cos 2t + 34 \sin 2t$\\
6. (C28) $y'' - 6y' - 7y = 8e^{-t} - 7t - 6$

\subsection*{Solution}
Assuming that the constants you find in your solution are all equal to 1, check your answer by calculating $y(t=0.4)$ in each case. To check your answer, please subsume all constants on any term into the constant that you set to 1. For example, instead of $y(t) = -2 C_1 e^{-t} + e^{-t}$, write $y(t) = C_1 e^{-t}$ where the two $e^{-t}$ terms have been combined and the $-2$ has been subsumed into the constant $C_1$, and then set $C_1 = 1$ to check the answer.




%%%%%%%%%%%%%%%%%%%%%%%%%%%%%%%%
\newpage
%%%%%%%%%%%%%%%%%%%%%%%%%%%%%%%%
\section{(C29) Method of undetermined coefficients: Determining the ODE I}

\subsection*{Comment}
This challenge gives you useful practise of going the other way; determining a differential equation that describes a given solution.

\subsection*{Challenge}
Determine the 2nd-order linear differential equation which has the general solution
\begin{equation}
    y = C_1 \cos 4t + C_2 \sin 4t - e^t \sin 2t
\end{equation}

\subsection*{Solution}
You will end up with differential terms on the left side and a function of $t$ on the right side.
Please compare your answer with your partner in class.
To check your answer with challenge-hub, evaluate the right side of your equation by substituting the value $t=0.4$.




%%%%%%%%%%%%%%%%%%%%%%%%%%%%%%%%
\newpage
%%%%%%%%%%%%%%%%%%%%%%%%%%%%%%%%
\section{(C30) Method of undetermined coefficients: Determining the ODE II}

\subsection*{Comment}
This challenge gives you useful practise of going the other way; determining a differential equation that describes a given solution.

\subsection*{Challenge}
Determine the 2nd-order linear differential equation which has the general solution
\begin{equation}
    y = C_1 e^{-2t} + C_2 t e^{-2t} + t^3 - 3t
\end{equation}

\subsection*{Solution}
You will end up with differential terms on the left side and a function of $t$ on the right side.
Please compare your answer with your partner in class.
To check your answer with challenge-hub, evaluate the right side of your equation by substituting the value $t=0.4$.

