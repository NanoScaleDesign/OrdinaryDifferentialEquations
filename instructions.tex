\section{This course}
This is the Autumn \courseyear \course course studied by \nensei-year undegraduate international students at Kyushu University.

\subsection{How this works}
\begin{itemize}
    \item In contrast to the traditional lecture-homework model, in this course the learning is self-directed and active via publicly-available resources.
    \item Learning is guided through solving a series of carefully-developed challenges contained in this book, coupled with suggested resources that can be used to solve the challenges with instant feedback about the correctness of your answer.
    \item There are no lectures. Instead, there is discussion time. Here, you are encouraged to discuss any issues with your peers, teacher and any teaching assistants. Furthermore, you are encouraged to help your peers who are having trouble understanding something that you have understood; by doing so you actually increase your own understanding too.
    \item Discussion-time is from \disctime on \discdays at room \discroom.
    \item Peer discussion is encouraged, however, if you have help to solve a challenge, always make sure you do understand the details yourself. You will need to be able to do this in an exam environment. If you need additional challenges to solidify your understanding, then ask the teacher. The questions on the exam will be similar in nature to the challenges. If you can do all of the challenges, you can get 100\% on the exam.
    \item Every challenge in the book typically contains a \textbf{Challenge} with suggested \textbf{Resources} which you are recommended to utilise in order to solve the challenge. Occasionally the teacher will provide extra \textbf{Comments} to help guide your thinking. A \textbf{Solution} is also made available for you to check your answer. Sometimes this solution will be given in encrypted form. For more information about encryption, see section \ref{sec:hashes}.
    \item For deep understanding, it is recommended to study the suggested resources beyond the minimum required to complete the challenge.
    \item The challenge document has many pages and is continuously being developed. Therefore it is advised to view the document on an electronic device rather than print it. The date on the front page denotes the version of the document. You will be notified by email when the document is updated. The content may differ from last-year's document.
    \item A target challenge will be set each week. This will set the pace of the course and defines the examinable material. It's ok if you can't quite reach the target challenge for a given week, but then you will be expected to make it up the next week.
    \item You may work ahead, even beyond the target challenge, if you so wish. This can build greater flexibility into your personal schedule, especially as you become busier towards the end of the semester.
    \item Your contributions to the course are strongly welcomed. If you come across resources that you found useful that were not listed by the teacher or points of friction that made solving a challenge difficult, please let the teacher know about it!
\end{itemize}

\subsection{Assessment}
In order to prove to outside parties that you have learned something from the course, we must perform summative assessments. This will be in the form of a mid-term exam (weighted 30\%), coursework (weighted 15\%), a satisfactory challenge-log (weighted 5\%) and a final exam (weighted 50\%).

Your final score is calculated as Max(final exam score, weighted score), however you must pass the final exam to pass the course.

\subsection{What you need to do}
\begin{itemize}
    \item Prepare a challenge-log in the form of a workbook or folder where you can clearly write the calculations you perform to solve each challenge. This will be a log of your progress during the course and will be occasionally reviewed by the teacher.
    \item You need to submit a brief report at \url{https://goo.gl/forms/AqTAZ6D1exFbH1PW2} by 4am on the day of the class. Here you can let the teacher know about any difficulties you are having and if you would like to discuss anything in particular.
    \item Please bring a wifi-capable internet device to class, as well as headphones if you need to access online components of the course during class. If you let me know in advance, I can lend computers and provide power extension cables for those who require them (limited number).
\end{itemize}
