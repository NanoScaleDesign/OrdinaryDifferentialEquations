\section{Sketching the system}
\label{sec:numericalfield}

\subsection*{Comment}
Numerical methods of solving ODE's are common. Here we will consider Euler and Runge-Kutta approaches.

\subsection*{Challenge}

Considering the following ODE:

\begin{equation}
    \dot{v} = g - v^2
\end{equation}

1. Sketch a direction field to show the behaviour of solutions to this ODE. In particular, what values of $v$ will lead to stable solutions?

2. What type of ODE is this?

\subsection*{Solutions}
1. (Stable)\\
\soltwodp{a}{4ac0c3}

1. (Unstable)\\
\soltwodp{b}{9ed192}

2. Please compare your solution with your partner or ask the teacher.




%%%%%%%%%%%%%%%%%%%%%%%%%%%%%%%%
\newpage
%%%%%%%%%%%%%%%%%%%%%%%%%%%%%%%%
\section{Tangent lines}

\subsection*{Resource}
\begin{itemize}
    \item Chapter 3: \url{https://raw.githubusercontent.com/kriskissel/ConceptsODE/master/main.pdf}
\end{itemize}

\subsection*{Challenge}
1. Given that $y(t)$ satisfies the equation $y' = y^3 + 3t$ subject to $y(1) = 2$, find $y'(1)$ without solving the differential equation and obtain the equation of the tangent to the curve $y(t)$ at the point (1,2).

2. Use the tangent line to estimte the value at $t = 1.5$.

\subsection*{Solution}
2. $7.5$




%%%%%%%%%%%%%%%%%%%%%%%%%%%%%%%%
\newpage
%%%%%%%%%%%%%%%%%%%%%%%%%%%%%%%%
\section{Euler's method}
\label{sec:euler}

\subsection*{Resource}
\begin{itemize}
    \item Chapter 3: \url{https://raw.githubusercontent.com/kriskissel/ConceptsODE/master/main.pdf}
\end{itemize}

\subsection*{Challenge}
Given that $v(t)$ satisfies the relation $v' = g - v^2$, assuming an initial value of $v(0)=0$, using Euler's method estimate $v(1)$ using step sizes of

1. $\Delta t = 1/2$\\
2. $\Delta t = 1/4$

Referring to the slope-field you drew in challenge \ref{sec:numericalfield}, explain the difference in the behaviour with the different step sizes. It may be helpful to draw a graph.

\subsection*{Solution}
1. $v(1) = -2.22$\\
2. $v(1) = 3.22$



%%%%%%%%%%%%%%%%%%%%%%%%%%%%%%%%
\newpage
%%%%%%%%%%%%%%%%%%%%%%%%%%%%%%%%
\section{4th-order Runge-Kutta}

\subsection*{Resource}
\begin{itemize}
    \item Chapter 3: \url{https://raw.githubusercontent.com/kriskissel/ConceptsODE/master/main.pdf}
\end{itemize}

\subsection*{Comment}
Derivation of the Runge-Kutta method is beyond this course, however there are many resources online going into more detail. This video-series is nice, although optional:
\begin{enumerate}
    \item \url{https://www.youtube.com/watch?v=b-OSyxOpxKc}
    \item \url{https://www.youtube.com/watch?v=JySrVHRmqfU}
    \item \url{https://www.youtube.com/watch?v=iS3hsHGY1Ok}
    \item \url{https://www.youtube.com/watch?v=wr3-dWoxiY4}
\end{enumerate}

\subsection*{Challenge}
1. Using the same function as challenge \ref{sec:euler}, estimate $v(1/2)$ using the Runge-Kutta method and a step-size of $\Delta t = \frac{1}{4}$.

2. Compare your answer to $v(1/2)$ obtained using the Euler method with the same step-size. How does the behaviour differ?

\subsection*{Solution}
1. $v(1/2) = 2.99$

2. Please compare your answer with your partner or check with the teacher.



\iffalse
%%%%%%%%%%%%%%%%%%%%%%%%%%%%%%%%
\newpage
%%%%%%%%%%%%%%%%%%%%%%%%%%%%%%%%
\section{Runge-Kutta vs Euler method}

\subsection*{Resource}
\begin{itemize}
    \item Chapter 3: \url{https://raw.githubusercontent.com/kriskissel/ConceptsODE/master/main.pdf}
\end{itemize}

\subsection*{Challenge}
Briefly explain the advantages that the Runge-Kutta method has over the Euler method.

\subsection*{Solution}
Please compare your answer with your partner or check with the teacher.
\fi
