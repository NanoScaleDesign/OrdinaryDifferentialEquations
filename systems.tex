\section{Homogeneous vs non-homogeneous}

\subsection*{Resources}
\begin{itemize}
    \item Page 1 of the PDF \url{http://www.math.psu.edu/tseng/class/Math251/Notes-LinearSystems.pdf}
\end{itemize}

\subsection*{Challenge}
Separately add the points of the following \emph{homogeneous} and \emph{non-homogeneous} ODE systems:

1 point:
$\displaystyle
\left(
    \begin{array}{c}
        x_1' \\
        x_2' \\
        x_3'
    \end{array}
\right)
=
\left(
    \begin{array}{ccc}
        1 & 2 & 3 \\
        4 & 5 & 6 \\
        7 & 8 & 9
    \end{array}
\right)
\left(
    \begin{array}{c}
        x_1 \\
        x_2 \\
        x_3
    \end{array}
\right)
+
\left(
    \begin{array}{c}
        0 \\
        0 \\
        0
    \end{array}
\right)
$

2 points:
$\displaystyle
\left(
    \begin{array}{c}
        x_1' \\
        x_2' \\
        x_3'
    \end{array}
\right)
=
\left(
    \begin{array}{ccc}
        1 & 2 & 3 \\
        4 & 5 & 6 \\
        7 & 8 & 9
    \end{array}
\right)
\left(
    \begin{array}{c}
        x_1 \\
        x_2 \\
        x_3
    \end{array}
\right)
+
\left(
    \begin{array}{c}
        Cos(t) \\
        0 \\
        0
    \end{array}
\right)$

4 points:
$\displaystyle
\left(
    \begin{array}{c}
        x_1' \\
        x_2' \\
        x_3'
    \end{array}
\right)
=
\left(
    \begin{array}{ccc}
        1 & 2 & 3 \\
        4 & 5 & 6 \\
        7 & 8 & 9
    \end{array}
\right)
\left(
    \begin{array}{c}
        x_1 \\
        x_2 \\
        x_3
    \end{array}
\right)
+
\left(
    \begin{array}{c}
        Cos(t) \\
        Sin(t) \\
        0
    \end{array}
\right)$

8 points:
$\displaystyle
\left(
    \begin{array}{c}
        x_1' \\
        x_2' \\
        x_3'
    \end{array}
\right)
=
\left(
    \begin{array}{ccc}
        1 & 2 & 3 \\
        4 & 5 & 6 \\
        7 & 8 & 9
    \end{array}
\right)
\left(
    \begin{array}{c}
        x_1 \\
        x_2 \\
        x_3
    \end{array}
\right)
+
\left(
    \begin{array}{c}
        Cos(t) \\
        Sin(t) \\
        Tan(t)
    \end{array}
\right)$

\subsection*{Solution}
Homogeneous: \hash{hhh}{106c67} \\
Non-homogeneous: \hash{iii}{d51f57}




%%%%%%%%%%%%%%%%%%%%%%%%%%%%%%%%
\newpage
%%%%%%%%%%%%%%%%%%%%%%%%%%%%%%%%
\section{Basis for creating a system of equations from a single ODE}

\subsection*{Resources}
\begin{itemize}
    \item Pages 1-4 of the PDF \url{http://www.math.psu.edu/tseng/class/Math251/Notes-LinearSystems.pdf} 
\end{itemize}

\subsection*{Comment}
Considering the general form of an nth-order linear equation,
\begin{equation}
    a_n y^{(n)} + a_{n-1} y^{(n-1)} + \cdots + a_1 y^1 + a_0 y = g(t)
\end{equation}
we substitute $x_1=y$, $x_2=y'$, \ldots, $x_n=y^{(n-1)}$ and $x_n'=y^{(n)}$.

When replacing a $y$-term by an $x$ term, the $n$ in $x_n$ corresponds to the number of times $y$ is differentiated, plus one. So $x_n$ corresponds to $y$ being differentiated $(n-1)$ times. Similarly, $x_n'$ corresponds to the differentiation one more time of $y^{(n-1)}$ (ie, $y^{(n)}$). Therefore, if the $y$-terms are replaced by $x'$ terms, then the $n$ in $x_n'$ corresponds to the number of times $y$ is differentiated (ie, $y^{(n)}$).

The examples on page 3 are clearer after reading page 4, so I encourage you to read page 4 before considering the examples.

Considering example (II) on page 3, you are given the equation
\begin{equation}
    y''' - 2y'' + 3y' - 4y = 0
\end{equation}

To add a more detailed explanation to that found in the PDF: First re-write the ODE in terms of $x$ and $x'$. Note that there is no ``$x_0'$'' so we just write it as $x_1$ in both equations.
\begin{align}
    x_4 - 2 x_3 + 3 x_2 - 4 x_1 &= 0 \label{eq:xs} \\
    x_3' - 2 x_2' + 3 x_1' - 4 x_1 &= 0 \label{eq:xprimes}
\end{align}

Our aim is to write the system of equations in the form $\bm{x'} = \bm{A}\bm{x}$. Note that there is no ``$x_4'$'' in our equations, so the largest value of $n$ in $x_n'$ will be 3 (ie, $x_3'$).
\begin{equation}
    \left(
        \begin{array}{c}
            x_1' \\
            x_2' \\
            x_3'
        \end{array}
    \right)
    =
    \left(
        \begin{array}{ccc}
            ? & ? & ? \\
            ? & ? & ? \\
            ? & ? & ?
        \end{array}
    \right)
    \left(
        \begin{array}{c}
            x_1 \\
            x_2 \\
            x_3
        \end{array}
    \right)
\end{equation}
where the question marks are values that we have to find.

By direct comparison of equations \ref{eq:xs} and \ref{eq:xprimes} we know that $x_1' = x_2$ which can be written as $x_1' = 0 x_1 + 1 x_2 + 0 x_3$ yielding the first line in the matrix $\bm{A}$:
\begin{equation}
    \left(
        \begin{array}{c}
            x_1' \\
            x_2' \\
            x_3'
        \end{array}
    \right)
    =
    \left(
        \begin{array}{ccc}
            0 & 1 & 0 \\
            ? & ? & ? \\
            ? & ? & ?
        \end{array}
    \right)
    \left(
        \begin{array}{c}
            x_1 \\
            x_2 \\
            x_3
        \end{array}
    \right)
\end{equation}

We can then proceed to do $x_2$ in a similar fashion:
\begin{equation}
    \left(
        \begin{array}{c}
            x_1' \\
            x_2' \\
            x_3'
        \end{array}
    \right)
    =
    \left(
        \begin{array}{ccc}
            0 & 1 & 0 \\
            0 & 0 & 1 \\
            ? & ? & ?
        \end{array}
    \right)
    \left(
        \begin{array}{c}
            x_1 \\
            x_2 \\
            x_3
        \end{array}
    \right)
\end{equation}

In order to express $x_3'$ in the above matrix form, we need it in terms of $x_1$, $x_2$ and $x_3$ rather than $x_4$, so instead of direct comparison, we swap $x_4$ for $x_3'$ in equation \ref{eq:xs} to read
\begin{equation}
    x_3' - 2 x_3 + 3 x_2 - 4 x_1 = 0
\end{equation}
and then isolate $x_3'$ to read $x_3' = 4 x_1 - 3 x_2 + 2 x_3$ yielding the final form of our systems of equations
\begin{equation}
    \left(
        \begin{array}{c}
            x_1' \\
            x_2' \\
            x_3'
        \end{array}
    \right)
    =
    \left(
        \begin{array}{ccc}
            0 & 1 & 0 \\
            0 & 0 & 1 \\
            4 & -3 & 2
        \end{array}
    \right)
    \left(
        \begin{array}{c}
            x_1 \\
            x_2 \\
            x_3
        \end{array}
    \right)
\end{equation}

Note that this is only considering a homogeneous equation. If it is non-homogeneous, you will have an extra term in the final step and will need a matrix of the form  $\bm{x'} = \bm{A}\bm{x} + \bm{g}$ as shown in the answer to exercise 4(b) on page 5 of the PDF.

So why do we want to do this? Well, notice that in this example we started with a complicated 3rd-order ODE and reduced it into 3 1st-order ODE's. Similarly, if we started with a 2nd-order ODE, we could reduce the equation to 2 1st-order ODE's. In general, for an nth-order ODE we can reduce it to $n$ 1st-order ODE's. If we can then learn how to solve simultanious sets of 1st-order ODE's, we have a powerful method of increasing our understanding (and even solving) difficult higher-order ODE's. 

Similarly, if you are given a system of 2 1st-order ODE's, you can know that it can form a single 2nd-order ODE. 

\subsection*{Challenge}
Write the following ODE's in matrix form:

1) $2 y'' + 4 y' - 6 y = 0$

2) $y'' + y = Cos(t)$

Complete exercises 1 and 2 on page 5 of the PDF.

\subsection*{Solutions}
To check your answers, sum the values of all the terms in your matrix $\bm{A}$.

1) 2

2) -1 (remember there is also a $+\bm{g}$ column-vector added to $\bm{A}\bm{x}$ too)

The answers to the PDF exercises are shown on page 5 of the PDF. Perhaps obviously, since you will not have the answers in a real-life/exam environment, please don't review each answer until completion. If you get stuck, be sure to review your notes (especially the worked-examples in the PDF) rather than the answers, to facilitate deep learning. 




%%%%%%%%%%%%%%%%%%%%%%%%%%%%%%%%
\newpage
%%%%%%%%%%%%%%%%%%%%%%%%%%%%%%%%
%\input{matricies}
\section{Matricies}

\subsection*{Resources}
\begin{itemize}
    \item PDF: Pages 6-17 of the PDF \url{http://www.math.psu.edu/tseng/class/Math251/Notes-LinearSystems.pdf}
\end{itemize}

\subsection*{Comment}
It is worth spending some time getting comfortable with manipulating matricies, since this is an indispensible basis for the work that is about to follow. The PDF gives a quick introduction to matricies. For a more thorough introduction, the Khan Academy playlist on linear algebra [1] is excellent, although beyond the scope of this course.

One note to deal with any confusion arising with regard to eigenvectors with matricies with zeros. For $(A-rI)$ equal to something like
\begin{equation}
\left(
    \begin{array}{cc}
        0 & 0 \\
        1 & 2
    \end{array}
\right)
\end{equation}
the top row can be ignored since any $x_1$ and $x_2$ will satisfy the top row.

Similarly, for a case such as
\begin{equation}
\left(
    \begin{array}{cc}
        2 & 0 \\
        2 & 0
    \end{array}
\right)
\end{equation}
you will have
\begin{align}
    2 x_1 + 0 x_2 &= 0 \\
    2 x_1 &= 0 \\
    x_1 &= 0
\end{align}
which is satisfied by
\begin{equation}
\left(
    \begin{array}{c}
        0 \\
        1
    \end{array}
\right)
\end{equation}
(where the $1$ could in principle be any number, but is the minimum integer that satisfies the condition.)

Finally, note that $(A-rI) = ((a, b), (c, d))$ will give you two equivalent formulas $a x_1 + b x_2 = 0$ and $c x_1 + b x_2 = 0$, even if they may appear different on first glance. If you want, you can prove to yourself that they are the same by multiplying the bottom row by $a/c$.

\vspace{0.2cm}
\noindent [1] \url{https://www.khanacademy.org/math/linear-algebra/alternate-bases}

\subsection*{Challenges}
Complete exersizes 1, 2, 3, 4 (I and II only) and 5 on page 18 of the PDF.

1. Calculate the eigenvectors and eigenvalues of
$\displaystyle
\left(
    \begin{array}{cc}
        -3 & 6 \\
        -3 & 3
    \end{array}
\right)$


\subsection*{Solutions}
The answers to the PDF exercises are shown on page 18 of the PDF. Perhaps obviously, since you will not have the answers in a real-life/exam environment, please don't review each answer until completion. If you get stuck, be sure to review your notes (especially the worked-examples in the PDF) rather than the answers, to facilitate deep learning. 

The solution to question 1 above can be found on the next page.

\newpage
1. Eigenvaluesof $\pm3i$. Eigenvectors of $(s,s(-1-i)/2)$ where $s$ is any multiplier (real or complex!) and $(s,s(-1+i)/2)$.




\iffalse
%%%%%%%%%%%%%%%%%%%%%%%%%%%%%%%%
\newpage
%%%%%%%%%%%%%%%%%%%%%%%%%%%%%%%%
\section{Solving systems of ODE's}

\subsection*{Resources}
\begin{itemize}
    \item Pages 6-31 of the PDF \url{http://www.math.psu.edu/tseng/class/Math251/Notes-LinearSystems.pdf} 
\end{itemize}

\subsection*{Challenge}
Complete at least exercises 1-5 on page 32-33 of the PDF.

\subsection*{Solutions}
The answers are shown on page 33-34 of the PDF. Perhaps obviously, since you will not have the answers in a real-life/exam environment, please don't review each answer until completion. If you get stuck, be sure to review your notes (especially the worked-examples in the PDF) rather than the answers, to facilitate deep learning. 
\fi
