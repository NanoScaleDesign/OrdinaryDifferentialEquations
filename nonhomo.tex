%%%%%%%%%%%%%%%%%%%%%%%%%%%%%%%%
\newpage
%%%%%%%%%%%%%%%%%%%%%%%%%%%%%%%%
\section{Non-homogeneous equations: Method of undetermined coefficients}

\subsection*{Resources}
\begin{itemize}
    \item Video: All 4 Khan Academy videos starting at \url{https://www.khanacademy.org/math/differential-equations/second-order-differential-equations/undetermined-coefficients/v/undetermined-coefficients-1}
    \item PDF: \url{http://www.math.psu.edu/tseng/class/Math251/Notes-2nd\%20order\%20ODE\%20pt2.pdf}
\end{itemize}

\subsection*{Comment}
The 2nd-order equations we were considering until now were homogeneous equations (ie, the RHS was zero). We can now build upon this to expand our ability to solve non-homogeneous equations (ie, where the RHS of the equation is non-zero). This will put you in a really strong position in terms of solving certain classes of 2nd-order ODE's.

The Khan Academy videos give an excellent initial introduction to the subject, and so please do take the time to view and take notes about all four videos in the series. The PDF listed above then allows us to develop our repertoire much further and explains very clearly about cases more-complicated than the videos. Please therefore make notes covering the material in the PDF from page 11 to onwards. Prior pages are largely covered by the videos.

You may note that in the PDF, the particular solution is denoted by $Y$ while Sal Khan denotes it as $y_p$ in the videos.

\subsection*{Challenge}
Complete questions 1-4 on page 22 in the PDF. These first challenges cover the fundamental basic cases upon which all subsequent cases are built.

\subsection*{Solution}
The questions in this challenge are taken from the PDF and the answers can be found on the last page. Perhaps obviously, since you will not have the answers in a real-life/exam environment, please don't review each answer until completion. If you get stuck, be sure to review your notes (especially the worked-examples in the PDF) rather than the answers, to facilitate deep learning. 

\timebox




%%%%%%%%%%%%%%%%%%%%%%%%%%%%%%%%
\newpage
%%%%%%%%%%%%%%%%%%%%%%%%%%%%%%%%
\section{Method of undetermined coefficients II}

\subsection*{Resources}
\begin{itemize}
    \item PDF: \url{http://www.math.psu.edu/tseng/class/Math251/Notes-2nd\%20order\%20ODE\%20pt2.pdf}
\end{itemize}

\subsection*{Challenge}
Continuing from the previous challenge, complete questions 5-10 on page 22 in the PDF. These challenges introduce a range of more complex situations and thus provide excellent practise of the concepts covered.

\subsection*{Solution}
The questions in this challenge are taken from the PDF and the answers can be found on the last page. Perhaps obviously, since you will not have the answers in a real-life/exam environment, please don't review each answer until completion. If you get stuck, be sure to review your notes (especially the worked-examples in the PDF) rather than the answers, to facilitate deep learning. 

\timebox




%%%%%%%%%%%%%%%%%%%%%%%%%%%%%%%%
\newpage
%%%%%%%%%%%%%%%%%%%%%%%%%%%%%%%%
\section{Method of undetermined coefficients III}

\subsection*{Resources}
\begin{itemize}
    \item PDF: \url{http://www.math.psu.edu/tseng/class/Math251/Notes-2nd\%20order\%20ODE\%20pt2.pdf}
\end{itemize}

\subsection*{Comment}
This challenge gives you useful practise of going the other way; determining a differential equation that describes a given solution. This can be a little confusing at first, so take time to understand where things originate from.

\subsection*{Challenge}
Complete challenges 19 and 20 from page 23 of the PDF.

\subsection*{Solution}
The questions in this challenge are taken from the PDF and the answers can be found on the last page. Perhaps obviously, since you will not have the answers in a real-life/exam environment, please don't review each answer until completion. If you get stuck, be sure to review your notes (especially the worked-examples in the PDF) rather than the answers, to facilitate deep learning. 


