\section{Order of a differential equation}

\subsection*{Resources}
\begin{itemize}
    \item Text: \url{http://tutorial.math.lamar.edu/Classes/DE/Definitions.aspx}
\end{itemize}

\subsection*{Challenge}
What is the sum of the orders of the following equations?

\begin{equation}
    \frac{dy}{dx}A = 5x^3 + 3
\end{equation}

\begin{equation}
    cos(y) y'''(x) - y(x) = 25
\end{equation}

\begin{equation}
    \frac{d}{dx} \frac{d^2 y}{dx^2} = \frac{x^{-2}}{3}
\end{equation}

\subsection*{Solution}
X = Your solution\\
Form: Integer\\
Place the indicated letter in front of the number\\
Example: aX where $X=46$ is entered as \href{http://www.wolframalpha.com/input/?i=md5+hash+of+\%22a46\%22}{a46}

hash of eX = 492585




%%%%%%%%%%%%%%%%%%%%%%%%%%%%%%%%%
\newpage
%%%%%%%%%%%%%%%%%%%%%%%%%%%%%%%%%
\section{Identifying linear and non-linear differential equations}

\subsection*{Comment}
Being able to identify linear and non-linear ODE's will help you understand how to approach different problems.

Generally speaking, the differential equation is linear if the functions and orders of the differentials are linear. For example, 

\begin{equation*}
    y'' - 4yx = ln x - y
\end{equation*}
can be shown to be linear. Rearranging to collect all the $y$-terms together:
\begin{equation*}
    y'' - 4yx + y = ln x
\end{equation*}
the dependent variable $y$ and its derivatives are each of the first degree and depend only on a constant or the independent variable.

An example of a non-linear equation however would be
\begin{equation*}
    5 + yy' = x - y
\end{equation*}
or
\begin{equation*}
    yy' + y = x - 5
\end{equation*}
The fact that $y'$ is multiplied by $y$ results in a non-linear equation in $y$.

\subsection*{Challenge}
Sum the points corresponding to the equations that are linear. You may be able to judge some by eye, but you should prove mathematically that at least one of the equations are linear and at least one of the equations are non-linear.

1 point: $\displaystyle \frac{dy}{dt} = 5t^3 + 3$.

2 points: $\displaystyle cos(y) y'''(t) - y(t) = 25$.

4 points: $\displaystyle \frac{d}{dt} \frac{d^2 y}{dt^2} = \frac{t^{-2}}{3}$.

8 points: $\displaystyle y'(t) - sin(y(t)) = 0$.

16 points: $\displaystyle y'(t) - y(t) = 0$.

32 points: $\displaystyle t y'(t) - y(t) = 0$.

\subsection*{Solution}
X = Your solution\\
Form: Integer\\
Place the indicated letter in front of the number\\
Example: aX where $X=46$ is entered as \href{http://www.wolframalpha.com/input/?i=md5+hash+of+\%22a46\%22}{a46}

hash of rX = f5d2c0




%%%%%%%%%%%%%%%%%%%%%%%%%%%%%%%%%
\newpage
%%%%%%%%%%%%%%%%%%%%%%%%%%%%%%%%%
\section{Linear differential equations vs non-linear differential equations}

\subsection*{Resources}
\begin{itemize}
    \item Wikipedia: \url{https://en.wikipedia.org/wiki/Nonlinear_system#Nonlinear_differential_equations}
    \item Wikipedia: \url{https://en.wikipedia.org/wiki/Linear_differential_equation}
\end{itemize}

\subsection*{Challenge}
Write no-more than 1 short paragraph describing in qualitative terms the difference between a linear and non-linear differential equation.

\subsection*{Solution}
Please compare with your partner in class and discuss with the teacher if you are unsure.




%%%%%%%%%%%%%%%%%%%%%%%%%%%%%%%%%
\newpage
%%%%%%%%%%%%%%%%%%%%%%%%%%%%%%%%%

\section{Valid solutions}

\subsection*{Resources}
\begin{itemize}
    \item Text: \url{http://tutorial.math.lamar.edu/Classes/DE/Definitions.aspx}
\end{itemize}

\subsection*{Challenge}

Use substitution to prove that

\begin{equation}
    y=\frac{5}{5+x}
\end{equation}

is a solution to the equation

\begin{equation}
    x y'+y=y^2
\end{equation}

and state the value of $x$ for which the solution is undefined.

\subsection*{Solution}
Value of $x$ for which solution is undefined:

\soltwodp{t}{829f33}




%%%%%%%%%%%%%%%%%%%%%%%%%%%%%%%%%
\newpage
%%%%%%%%%%%%%%%%%%%%%%%%%%%%%%%%%

\section{Range of valid solutions}

\subsection*{Resources}
\begin{itemize}
    \item Text: \url{http://tutorial.math.lamar.edu/Classes/DE/Definitions.aspx}
\end{itemize}

\subsection*{Challenge}

Use substitution to prove that

\begin{equation}
    y = -\sqrt{100-x^2}
\end{equation}

is a solution to the equation

\begin{equation}
    x + y y' = 0
\end{equation}

and state the range of x for which the solution is valid. Enter the value of the lower range as the solution below.

\subsection*{Solution}
\soltwodp{y}{d96920}
