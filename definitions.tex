\section{Order of a differential equation}

\subsection*{Resources}
\begin{itemize}
    \item Text: \url{http://tutorial.math.lamar.edu/Classes/DE/Definitions.aspx}
\end{itemize}

\subsection*{Challenge}
What is the sum of the orders of the following equations?

\begin{equation}
    \frac{dy}{dx}A = 5x^3 + 3
\end{equation}

\begin{equation}
    cos(y) y'''(x) - y(x) = 25
\end{equation}

\begin{equation}
    \frac{d}{dx} \frac{d^2 y}{dx^2} = \frac{x^{-2}}{3}
\end{equation}

\subsection*{Solution}
\six{}

\hash{e}{adb5a9}

\timebox


%%%%%%%%%%%%%%%%%%%%%%%%%%%%%%%%%
\newpage
%%%%%%%%%%%%%%%%%%%%%%%%%%%%%%%%%

\section{Linear equations}

\subsection*{Resources}
\begin{itemize}
    \item Video: \url{https://www.youtube.com/watch?v=dwNMEMOGO_o}
    \item Text: \url{http://www.myphysicslab.com/classify_diff_eq.html}
    \item Text: \url{http://tutorial.math.lamar.edu/Classes/DE/Definitions.aspx}
\end{itemize}

\subsection*{Comment}
Using a function $x$ dependent on time $t$ as an example, a differential equation is defined as being linear if it can be written in the form

\begin{equation}
    f_n(t) \frac{d^nx}{d t^n} + \dots + f_1(t) \frac{dx}{dt} + a_0(t) x + C = 0
\end{equation}

Here, $f_n(t)$ is a function of time only, such as $5t$ or $2/t^2$ or may even be constant with time (eg, $3$).
Any of the $f_n$'s and the constant $C$ may be zero.
If it is possible to arrange an equation into the above form, then the equation must be linear.
So for a linear equation, $x(t)$, $x'(t)$, $t x(t)$, $3t^2 x'''(t)$ are linear terms in $x$, but $x(t)^2$, $x(t) x'(t)$ and $5 t Tan(x)$ are non-linear terms.

\subsection*{Challenge}
Sum the points corresponding to the equations that are linear:

1 point: $\displaystyle \frac{dx}{dt} = 5t^3 + 3$.

2 points: $\displaystyle cos(x) x'''(t) - x(t) = 25$.

4 points: $\displaystyle \frac{d}{dt} \frac{d^2 x}{dt^2} = \frac{t^{-2}}{3}$.

8 points: $\displaystyle x'(t) - sin(x(t)) = 0$.

16 points: $\displaystyle x'(t) - x(t) = 0$.

32 points: $\displaystyle t x'(t) - x(t) = 0$.

\subsection*{Solution}
\six{}

\hash{r}{9aea7d}

\timebox


%%%%%%%%%%%%%%%%%%%%%%%%%%%%%%%%%
\newpage
%%%%%%%%%%%%%%%%%%%%%%%%%%%%%%%%%

\section{Valid solutions}

\subsection*{Resources}
\begin{itemize}
    \item Text: \url{http://tutorial.math.lamar.edu/Classes/DE/Definitions.aspx}
\end{itemize}

\subsection*{Challenge}

Use substitution to prove that

\begin{equation}
    y=\frac{5}{5+x}
\end{equation}

is a solution to the equation

\begin{equation}
    x y'+y=y^2
\end{equation}

and state the value of $x$ for which the solution is undefined.

\subsection*{Solution}
Value of $x$ for which solution is undefined: \six{}

\hash{t}{c69a20}

\timebox


%%%%%%%%%%%%%%%%%%%%%%%%%%%%%%%%%
\newpage
%%%%%%%%%%%%%%%%%%%%%%%%%%%%%%%%%

\section{Range of valid solutions}

\subsection*{Resources}
\begin{itemize}
    \item Text: \url{http://tutorial.math.lamar.edu/Classes/DE/Definitions.aspx}
\end{itemize}

\subsection*{Challenge}

Use substitution to prove that

\begin{equation}
    y = -\sqrt{100-x^2}
\end{equation}

is a solution to the equation

\begin{equation}
    x + y y' = 0
\end{equation}

and state the range of x for which the solution is valid. Enter the value of the lower range as the solution below.

\subsection*{Solution}
\six{}

\hash{y}{ef0eb9}

\timebox
