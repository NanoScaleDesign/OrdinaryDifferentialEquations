\section{Order of a differential equation}

\subsection*{Resources}
\begin{itemize}
    \item Text: \url{http://tutorial.math.lamar.edu/Classes/DE/Definitions.aspx}
\end{itemize}

\subsection*{Challenge}
What is the sum of the orders of the following equations?

\begin{equation}
    \frac{dy}{dx}A = 5x^3 + 3
\end{equation}

\begin{equation}
    cos(y) y'''(x) - y(x) = 25
\end{equation}

\begin{equation}
    \frac{d}{dx} \frac{d^2 y}{dx^2} = \frac{x^{-2}}{3}
\end{equation}

\subsection*{Solution}
X = Your solution\\
Form: Integer\\
Place the indicated letter in front of the number\\
Example: aX where $X=46$ is entered as \href{http://www.wolframalpha.com/input/?i=md5+hash+of+\%22a46\%22}{a46}

hash of eX = 492585




%%%%%%%%%%%%%%%%%%%%%%%%%%%%%%%%%
\newpage
%%%%%%%%%%%%%%%%%%%%%%%%%%%%%%%%%
\section{Identifying linear and non-linear differential equations}

\subsection*{Comment}
Being able to identify linear and non-linear ODE's will help you understand how to approach different problems.

A function is linear if it satisfies the following condition for all $y_1$ and $y_2$:
\begin{equation}
    \label{eq:lindef}
    f(y_1 + y_2)  = f(y_1) + f(y_2)
\end{equation}

For example,
\begin{equation*}
    y = 5x^2 + 4
\end{equation*}
is trivally linear, because setting $f(y)$ to be equal to the LHS (which is only a function of $y$),
\begin{equation*}
    f(y_1 + y_2) = (y_1+y_2) = y_1 + y_2 = f(y_1) + f(y_2)
\end{equation*}

The function on the left side does not have to be so simple. It just has to be a function of $y$. For example, the equation
\begin{equation*}
    y'' - 4yx = ln x - y
\end{equation*}
is more complex. However it is still linear. Rearranging to collect all the $y$-terms together:
\begin{equation*}
    f(y) = y'' - 4yx + y = ln x
\end{equation*}
linearity can be proved:
\begin{align*}
    f(y_1 + y_2) &= (y_1 + y_2)'' - 4(y_1 + y_2)x + (y_1 + y_2) \\
                 &= y_1''+ y_2'' - 4y_1x + 4y_2x + y_1 + y_2 \\
                 &= y_1 + 3y_1x + y_1 + y_2 + 3y_2x + y_2 \\
                 &= f(y_1) + f(y_2)
\end{align*}

For non-linear equations however, equation \ref{eq:lindef} does not hold true. For example,
\begin{equation*}
    5 + yy' = x - y
\end{equation*}
can be rearranged to isolate all the y-terms on the LHS as
\begin{equation*}
    f(y) = yy' + y = x - 5
\end{equation*}
however it can easily be seen than \ref{eq:lindef} does not hold true:
\begin{align*}
    f(y_1 + y_2) &= (y_1 + y_2)(y_1+y_2)' + (y_1 + y_2) \\
                 &= y_1 y_1' + y_2 y_2' + y_1 y_2' + y_2 y_1' + y_1 + y_2 \\
                 &= y_1 y_1' + y_1 + y_2 y_2' + y_2 + y_1 y_2' + y_2 y_1' \\
                 &= f(y_1) + f(y_2) + y_1 y_2' + y_2 y_1'
\end{align*}
which does not satisfy equation \ref{eq:lindef} for all $y_1$ and $y_2$.

\subsection*{Challenge}
Sum the points corresponding to the equations that are linear. You may be able to judge some by eye, but you should prove mathematically that at least one of the equations are linear and at least one of the equations are non-linear.

1 point: $\displaystyle \frac{dx}{dt} = 5t^3 + 3$.

2 points: $\displaystyle cos(x) x'''(t) - x(t) = 25$.

4 points: $\displaystyle \frac{d}{dt} \frac{d^2 x}{dt^2} = \frac{t^{-2}}{3}$.

8 points: $\displaystyle x'(t) - sin(x(t)) = 0$.

16 points: $\displaystyle x'(t) - x(t) = 0$.

32 points: $\displaystyle t x'(t) - x(t) = 0$.

\subsection*{Solution}
X = Your solution\\
Form: Integer\\
Place the indicated letter in front of the number\\
Example: aX where $X=46$ is entered as \href{http://www.wolframalpha.com/input/?i=md5+hash+of+\%22a46\%22}{a46}

hash of rX = f5d2c0




%%%%%%%%%%%%%%%%%%%%%%%%%%%%%%%%%
\newpage
%%%%%%%%%%%%%%%%%%%%%%%%%%%%%%%%%
\section{Linear differential equations vs non-linear differential equations}

\subsection*{Resources}
\begin{itemize}
    \item Wikipedia: \url{https://en.wikipedia.org/wiki/Nonlinear_system#Nonlinear_differential_equations}
    \item Wikipedia: \url{https://en.wikipedia.org/wiki/Linear_differential_equation}
\end{itemize}

\subsection*{Challenge}
Write no-more than 1 short paragraph describing in qualitative terms the difference between a linear and non-linear differential equation.

\subsection*{Solution}
Please compare with your partner in class and discuss with the teacher if you are unsure.




%%%%%%%%%%%%%%%%%%%%%%%%%%%%%%%%%
\newpage
%%%%%%%%%%%%%%%%%%%%%%%%%%%%%%%%%

\section{Valid solutions}

\subsection*{Resources}
\begin{itemize}
    \item Text: \url{http://tutorial.math.lamar.edu/Classes/DE/Definitions.aspx}
\end{itemize}

\subsection*{Challenge}

Use substitution to prove that

\begin{equation}
    y=\frac{5}{5+x}
\end{equation}

is a solution to the equation

\begin{equation}
    x y'+y=y^2
\end{equation}

and state the value of $x$ for which the solution is undefined.

\subsection*{Solution}
Value of $x$ for which solution is undefined:

\soltwodp{t}{829f33}




%%%%%%%%%%%%%%%%%%%%%%%%%%%%%%%%%
\newpage
%%%%%%%%%%%%%%%%%%%%%%%%%%%%%%%%%

\section{Range of valid solutions}

\subsection*{Resources}
\begin{itemize}
    \item Text: \url{http://tutorial.math.lamar.edu/Classes/DE/Definitions.aspx}
\end{itemize}

\subsection*{Challenge}

Use substitution to prove that

\begin{equation}
    y = -\sqrt{100-x^2}
\end{equation}

is a solution to the equation

\begin{equation}
    x + y y' = 0
\end{equation}

and state the range of x for which the solution is valid. Enter the value of the lower range as the solution below.

\subsection*{Solution}
\soltwodp{y}{d96920}
