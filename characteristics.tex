\section{Free vibration of a spring (review)}

\subsection*{Challenge}
Consider a mass on a spring undergoing vibration with no damping or external forcing, just as you considered in challenge \ref{sec:hooke}.

Show that the position of the mass as a function of time can be given by
\begin{equation}
    \label{eq:freevib}
    x(t) = C_1 \sin \omega_0 t + C_2 \cos \omega_0 t
\end{equation}
where $C_1$ and $C_2$ are constants. 

\subsection*{Solution}
Please discuss with the teacher or your peers if you have any difficulty.




%%%%%%%%%%%%%%%%%%%%%%%%%%%%%%%%
\newpage
%%%%%%%%%%%%%%%%%%%%%%%%%%%%%%%%
\section{Phase-shift}

\subsection*{Comment}
Using the trigonometric identity
\begin{equation}
    \cos(\alpha \pm \beta) = \cos \alpha \cos \beta \mp \sin \alpha \sin \beta
\end{equation}
it is possible to write equation \ref{eq:freevib} as
\begin{equation}
    x(t) = C_3 \cos(\omega_0 t - \phi)
\end{equation}
where $C_1 = C_3 \cos \phi$ and $C_2 = C_3 \sin \phi$, while the magnitude of $C_3$ can be found using $C_3^2 = C_1^2 + C_2^2$.

\subsection*{Challenge}
Re-writing sine as a phase-shifted cosine; ie, $\sin \omega_0 t$ becoming $\cos (\omega_0 t - \phi)$; if $\omega_0 = 1$, what is $\phi$ (use the lowest-possible positive solution)?

\subsection*{Solution}
\soltwodp{f}{f97726}




%%%%%%%%%%%%%%%%%%%%%%%%%%%%%%%%
\newpage
%%%%%%%%%%%%%%%%%%%%%%%%%%%%%%%%
\section{Derivation of a periodically-forced system}

\subsection*{Challenge}
Now consider that there is some external forcing of the form $F \cos \omega t$ where $F$ is some constant.
Note that $\omega$ is the frequency at which the forcing varies, and $\omega_0^2 = k/m$ is the ``natural frequency'' of the unforced system.

Assuming $\omega \ne \omega_0$, show that the position of the forced system as a function of time can be given by
\begin{equation}
    \label{eq:forcedpostime}
    x(t) = C_3 \cos(\omega_0 t - \phi) + \frac{F}{m(\omega_0^2 - \omega^2)} \cos \omega t
\end{equation}




%%%%%%%%%%%%%%%%%%%%%%%%%%%%%%%%
\newpage
%%%%%%%%%%%%%%%%%%%%%%%%%%%%%%%%
\section{Beating equation}

\subsection*{Challenge}
\emph{Assuming that the mass on the spring starts at rest at position ``0''}, solve for the constant $C_3$ and the phase-shift $\phi$, then use trigonometric identity
\begin{equation}
    \cos \alpha - \cos \beta = 2 \sin \left( \frac{\alpha - \beta}{2} t \right) \sin \left( \frac{\alpha + \beta}{2} t \right)
\end{equation}
to show that the position of the mass as described in equation \ref{eq:forcedpostime} can be given by
\begin{equation}
    x(t) = \frac{2 F}{m(\omega_0^2 - \omega^2)} \sin \left( \frac{\omega - \omega_0}{2} t \right) \sin \left( \frac{\omega + \omega_0}{2} t \right)
\end{equation}




%%%%%%%%%%%%%%%%%%%%%%%%%%%%%%%%
\newpage
%%%%%%%%%%%%%%%%%%%%%%%%%%%%%%%%
\section{Beating explanation}

\section*{Resources}
\begin{itemize}
    \item \url{https://www.johndcook.com/blog/2013/02/22/undamped-forced-vibrations/}
    \item Video: \url{https://www.youtube.com/watch?v=v3ImPthjI3o}
\end{itemize}

\subsection*{Comment}
The equation that you obtained in the previous challenge demonstrates nicely the phenomenon of beating. You have a high-frequency wave multiplied by a low frequency wave. The image in the above resource depicts this visually very nicely and you can also hear a representation of this in sound with the video resource above.

\subsection*{Challenge}
Draw a graph depicting what happens as time evolves, and briefly explain what ``beating'' is and how it arises.




%%%%%%%%%%%%%%%%%%%%%%%%%%%%%%%%
\newpage
%%%%%%%%%%%%%%%%%%%%%%%%%%%%%%%%
\section{Resonance}

\section*{Resources}
\begin{itemize}
    \item \url{https://www.johndcook.com/blog/2013/02/22/undamped-forced-vibrations/}
    \item Video: \url{https://www.youtube.com/watch?v=v3ImPthjI3o}
\end{itemize}

\subsection*{Challenge}
Resonance occurs when the system is forced at its natural frequency (ie, $\omega = \omega_0$).

1. Show that when starting from position ``0'' at rest, when the system is driven at its natural frequency it evolves with time according to the equation
\begin{equation}
    x(t) = \frac{F}{2 m \omega_0} t \sin \omega t
\end{equation}

2. Draw a graph depicting what is happening as time evolves, and briefly explain what resonance is.
